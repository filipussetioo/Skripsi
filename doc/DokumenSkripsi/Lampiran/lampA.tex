%versi 3 (18-12-2016)
\chapter{Kode Program}
\label{lamp:A}

%terdapat 2 cara untuk memasukkan kode program
% 1. menggunakan perintah \lstinputlisting (kode program ditempatkan di folder yang sama dengan file ini)
% 2. menggunakan environment lstlisting (kode program dituliskan di dalam file ini)
% Perhatikan contoh yang diberikan!!
%
% untuk keduanya, ada parameter yang harus diisi:
% - language: bahasa dari kode program (pilihan: Java, C, C++, PHP, Matlab, C#, HTML, R, Python, SQL, dll)
% - caption: nama file dari kode program yang akan ditampilkan di dokumen akhir
%
% Perhatian: Abaikan warning tentang textasteriskcentered!!
%

Kode perangkat lunak hanya akan dilampirkan beberapa \textit{file} dengan perubahan terbanyak. Lampiran kode lainnya tidak ditampilkan karena dapat memperpanjang dan memperberat dokumen tugas akhir ini. Lampiran lainnya dapat dilihat pada \url{https://github.com/filipussetioo/SharIF-JudgeCI4}.

\section{\textit{Controller}}
\label{sec:lampContr}
\lstinputlisting[language=PHP, caption=Assignments.php]{./Lampiran/controller/Assignments.php}
\lstinputlisting[language=PHP, caption=BaseController.php]{./Lampiran/controller/BaseController.php}
\lstinputlisting[language=PHP, caption=Submissions.php]{./Lampiran/controller/Submissions.php}
\lstinputlisting[language=PHP, caption=Submit.php]{./Lampiran/controller/Submit.php}

\section{\textit{Model}}
\label{sec:lampModel}
\lstinputlisting[language=PHP, caption=AssignmentModel.php]{./Lampiran/model/AssignmentModel.php}
\lstinputlisting[language=PHP, caption=ScoreboardModel.php]{./Lampiran/model/ScoreboardModel.php}
\lstinputlisting[language=PHP, caption=UserModel.php]{./Lampiran/model/UserModel.php}

\section{\textit{View}}
\label{sec:lampView}
\lstinputlisting[language=PHP, caption=add\_assignment.php]{./Lampiran/view/add_assignment.php}
\lstinputlisting[language=PHP, caption=settings.php]{./Lampiran/view/settings.php}
\lstinputlisting[language=PHP, caption=assignments.php]{./Lampiran/view/assignments.php}
\lstinputlisting[language=PHP, caption=dashboard.php]{./Lampiran/view/dashboard.php}
\lstinputlisting[language=PHP, caption=problems.php]{./Lampiran/view/problems.php}
\lstinputlisting[language=PHP, caption=base.php]{./Lampiran/view/base.php}

\section{Kode lainnya}
\label{sec:lampLain}
Berikut merupakan lampiran kode lainnya seperti \texttt{Filters}, aturan \texttt{Validation}, dan lainnya.

\lstinputlisting[language=PHP, caption=CheckInstallAndLogin.php]{./Lampiran/others/CheckInstallAndLogin.php}
\lstinputlisting[language=PHP, caption=CheckLoginandisAjax.php]{./Lampiran/others/CheckLoginandisAjax.php}
\lstinputlisting[language=PHP, caption=CheckLoginandLevelAdmin.php]{./Lampiran/others/CheckLoginandLevelAdmin.php}
\lstinputlisting[language=PHP, caption=Validation.php]{./Lampiran/others/Validation.php}

\iffalse
\begin{lstlisting}[language=Java, caption=MyCode.c]

// This does not make algorithmic sense, 
// but it shows off significant programming characters.

#include<stdio.h>

void myFunction( int input, float* output ) {
	switch ( array[i] ) {
		case 1: // This is silly code
			if ( a >= 0 || b <= 3 && c != x )
				*output += 0.005 + 20050;
			char = 'g';
			b = 2^n + ~right_size - leftSize * MAX_SIZE;
			c = (--aaa + &daa) / (bbb++ - ccc % 2 );
			strcpy(a,"hello $@?"); 
	}
	count = ~mask | 0x00FF00AA;
}

// Fonts for Displaying Program Code in LATEX
// Adrian P. Robson, nepsweb.co.uk
// 8 October 2012
// http://nepsweb.co.uk/docs/progfonts.pdf

\end{lstlisting}

\lstinputlisting[language=Java, caption=MyCode.java]{./Lampiran/MyCode.java} 

\fi