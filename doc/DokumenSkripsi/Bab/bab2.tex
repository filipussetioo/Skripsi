%versi 3 (22-07-2020)
\chapter{Landasan Teori}
\label{chap:teori}

\section{CodeIgniter 3 \cite{ci3:22}}
\label{sec:ci3} 
 
CodeIgniter 3 merupakan sebuah \textit{framework opensource} yang berfungsi untuk mempermudah pengguna dalam membangun aplikasi \textit{website} menggunakan bahasa PHP. \textit{CodeIgniter 3} juga merupakan \textit{framework} ringan yang menggunakan struktur \textit{Model-View-Controller}, dan menghasilkan URL yang bersih. CodeIgniter 3 memiliki tujuan untuk membantu pengguna dalam membangun aplikasi web lebih cepat dengan menyediakan beragam \textit{library} dan tampilan dan \textit{logic} yang simpel. \textit{Code Igniter 3} memiliki \textit{flow chart} aplikasi yang ditunjukan pada Gambar \ref{fig:ci3flowchart}.

\begin{figure}[H]
	\centering  
	\includegraphics[scale=0.5]{ci3flowchart}  
	\caption[\textit{Flow Chart} Aplikasi \textit{CodeIgniter 3}]{\textit{Flow Chart} Aplikasi \textit{CodeIgniter 3}} 
	\label{fig:ci3flowchart} 
\end{figure} 

Berikut merupakan pembagian \textit{flow chart} aplikasi \textit{CodeIgniter 3} seperti yang ditunjukan pada Gambar \ref{fig:ci3flowchart}:
\begin{enumerate}
\item \texttt{index.php} berfungsi sebagai \textit{front controller} yang berguna untuk melakukan inisiasi
\item \textit{Router} berfungsi untuk melakukan pemeriksaan dan menentukan penggunaan \textit{HTTP Request}.
\item \textit{Cache} berfungsi untuk mengirimkan \textit{file cache} kepada \textit{browser} secara langsung hanya jika terdapat \textit{cache}.   
\item \textit{Security} berfungsi untuk alat penyaringan setiap data dan \textit{HTTP Request} yang masuk. Penyaringan data tersebut dilakukan sebelum \textit{controller} aplikasi dimuat agar aplikasi menjadi lebih aman.
\item \textit{Controller} befungsi sebagai alat untuk memuat \textit{model, libraries}, dan sumber daya yang dibutuhkan untuk menjalankan permintaan spesifik.
\item \textit{View} dikirimkan menuju \textit{browser} untuk dilihat oleh pengguna. Apabila \textit{caching} dinyalakan, maka \textit{view} akan dilakukan \textit{cached} terlebih dahulu sehingga permintaan selanjutnya dapat diberikan.
\end{enumerate}

\subsection{\textit{Model-View-Controller}}
\textit{CodeIgniter 3} merupakan \textit{framework} berbasis arsitektur \textit{Model-View-Controller} atau yang selanjutnya akan disebut sebagai MVC. MVC merupakan sebuah pendekatan perangkat lunak yang memisahkan antara logika dengan presentasi atau tampilannya. Penggunaan struktur ini mengurangi penggunaan skrip pada halaman web karena tampilan terpisah dengan skrip PHP. Berikut merupakan penjelasan mengenai struktur MVC:

\subsubsection{\textit{Model}} \texttt{Model} berfungsi untuk mewakili struktur data perangkat lunak dalam mengambil, memasukan, dan memperbarui data pada \textit{database}. \textit{Model} biasanya digunakan pada \textit{file controller} dan dapat dipanggil menggunakan kode sebagai berikut: 
\begin{center}
\verb|$this->load->model('model_name');|.
\end{center}. 
Sintaks diatas akan memuat \texttt{model} dengan nama \texttt{model\_name}. Selanjutnya, pengguna dapat membentuk fungsi pada \textit{file} \texttt{model} sesuai dengan kebutuhan. Kode \ref{kode:model} menunjukan contoh \textit{file Model} \textit{CodeIgniter 3} pada direktori \verb|application/models/|:

\begin{lstlisting}[language=PHP, caption=Contoh \textit{model} pada \textit{CodeIgniter 3}, label=kode:model]
class Blog_model extends CI_Model {

        public $title;
        public $content;
        public $date;

        public function get_last_ten_entries()
        {
                $query = $this->db->get('entries', 10);
                return $query->result();
        }

        public function insert_entry()
        {
                $this->title    = $_POST['title']; // please read the below note
                $this->content  = $_POST['content'];
                $this->date     = time();

                $this->db->insert('entries', $this);
        }

        public function update_entry()
        {
                $this->title    = $_POST['title'];
                $this->content  = $_POST['content'];
                $this->date     = time();

                $this->db->update('entries', $this, array('id' => $_POST['id']));
        }

}
\end{lstlisting}

Berikut merupakan penjelasan beberapa fungsi pada Kode \ref{kode:model}:
\begin{itemize}
\item \verb|get_last_ten_entries()| yang berfungsi untuk mengambil 10 data terakhir dari tabel \verb|entries| menggunakan \textit{query builder}.
\item \verb|insert_entry()| yang berfungsi untuk memasukan data \textit{title, content}, dan \textit{date} menuju tabel \verb|entries|.
\item \verb|update_entry()| yang befungsi untuk memperbaharui data \textit{title, content}, dan \textit{date} pada tabel \verb|entries|.
\end{itemize}

\subsubsection{\textit{\textbf{Views}}}
\textit{View} berfungsi dalam menyajikan informasi kepada pengguna. \textit{View} biasanya merupakan halaman web, namun pada \textit{CodeIgniter 3} \textit{View} dapat berupa pecahan halaman seperti \textit{header}, \textit{footer}, \textit{sidebar}, dan lainnya. Pecahan halaman dapat dimasukan secara fleksibel apabila dibutuhkan.

\begin{lstlisting}[language=PHP, caption=Contoh \textit{view} pada \textit{CodeIgniter 3}, label=kode:view]
<?php
<html>
<head>
        <title>My Blog</title>
</head>
<body>
        <h1>Welcome to my Blog!</h1>
</body>
</html>
\end{lstlisting}

Kode \ref{kode:view} merupakan contoh \textit{file view} \textit{CodeIgniter 3} pada direktori \verb|application/views/| yang berisi judul \texttt{My Blog} dan \textit{heading} \texttt{Welcome to my Blog!}. \textit{View} tidak dapat dipanggil secara langsung, namun pengguna dapat memanggil halaman yang sudah dibentuk pada \textit{file controller} dengan cara sebagai berikut:

\begin{center}
\verb|$this->load->view('name');|
\end{center}

Sintaks diatas akan mengembalikan halaman \textit{view} dengan nama \texttt{name} dan menampilkannya kepada pengguna.

\subsubsection{\textit{\textbf{Controller}}} 
\textit{Controller} berfungsi sebagai perantara antara \textit{Model}, \textit{View}, dan sumber daya yang dibutuhkan untuk melakukan proses \textit{HTTP Request} dan menjalankan halaman web. Penamaan \textit{controller} biasanya digunakan sebagai \textit{url} pada perangkat lunak pengguna. Contoh \textit{controller} \textit{CodeIgniter 3} pada direktori \verb|application/controllers/| ditunjukan pada Kode \ref{kode:controller}:

\begin{lstlisting}[language=PHP, caption=Contoh \textit{controller} pada \textit{CodeIgniter 3}, label=kode:controller]
<?php
class Blog extends CI_Controller {

        public function index()
        {
                echo 'Hello World!';
        }

        public function comments()
        {
                echo 'Look at this!';
        }
}
\end{lstlisting}

Kode \ref{kode:controller} berfungsi dalam mengembalikan \textit{string} sesuai dengan fungsi \textit{controller} yang dipanggil. Nama \textit{controller} dan metode diatas akan dijadikan segmen pada \textit{URL} seperti berikut:

\begin{center}
\verb|example.com/index.php/blog/index/|
\end{center}

Metode \textit{index} akan secara otomatis dipanggil menjadi \textit{URL} dan pengguna juga dapat memberi parameter untuk metode \textit{controller} yang selanjutnya dapat diambil dari \textit{URL}.

\subsection{\textit{CodeIgniter URLs}}

\textit{CodeIgniter 3} menggunakan pendekatan \textit{segment-based} dibandingkan menggunakan \textit{query string} untuk membangun \textit{URL} yang mempermudah mesin pencari dan pengguna. Berikut merupakan contoh \textit{URL} pada \textit{CodeIgniter 3}:

\begin{center}
\verb|example.com/news/article/my_article|
\end{center}

\textit{URL} diatas dibentuk berdasarkan segmen \textit{Controller} sebagai berikut :

\begin{center}
\verb|example.com/class/function/ID|
\end{center}

Segmen tersebut dibagi menjadi tiga buah sebagai berikut:
\begin{enumerate}
\item Segmen pertama mereprentasikan kelas \textit{controller} yang dipanggil.
\item Segmen kedua mereprentasikan kelas fungsi atau metode yang digunakan.
\item Segmen ketiga dan segmen lainnya mereprentasikan \textit{ID} dari variabel yang akan dipindahkan menuju \textit{controller}.
\end{enumerate}

Secara \textit{default}, \textit{URL} yang dihasilkan \textit{CodeIgniter 3} terdapat nama \textit{file} \verb|index.php| seperti contoh dibawah ini:

\begin{center}
\verb|example.com/index.php/news/article/my_article|
\end{center}

Pengguna dapat menghapus \textit{file} \verb|index.php| \textit{file} pada \textit{url} menggunakan \textit{file} \verb|.htaccess| apabila \textit{server Apache} pengguna menghidupkan \verb|mod_rewrite|. Contoh \textit{file} \verb|.htaccess| menggunakan metode \textit{negative} ditunjukan pada Kode \ref{kode:htaccess}:

\begin{lstlisting}[caption=Contoh \textit{file} \texttt{.htaccess} pada halaman index.php, label=kode:htaccess]
RewriteEngine On
RewriteCond %{REQUEST_FILENAME} !-f
RewriteCond %{REQUEST_FILENAME} !-d
RewriteRule ^(.*)$ index.php/$1 [L]
\end{lstlisting}

Aturan pada Kode \ref{kode:htaccess} menyebabkan \textit{HTTP Request} selain yang berasal dari direktori atau \textit{file} diperlakukan sebagai sebuah permintaan pada \textit{file} \verb|index.php|. Selain itu, pengguna juga dapat menambahkan akhirkan pada URL agar halaman pengguna dapat menampilkan halaman sesuai dengan tipe yang diinginkan. Kode \ref{kode:htaccesssuffix} menunjukan contoh \textit{URL} sebelum dan sesudah ditambahkan akhiran berupa \texttt{.html}: 

\begin{lstlisting}[caption=Contoh URL sebelum dan sesudah ditambahkan akhiran, label=kode:htaccesssuffix]
example.com/index.php/products/view/shoes

example.com/index.php/products/view/shoes.html
\end{lstlisting}

Kode \ref{kode:htaccesssuffix} merupakan contoh URL sebelum dan sesudah ditambahkan akhiran. Pengguna juga dapat menyalakan fitur \textit{query strings} dengan cara sebagai mengubah \textit{file} \texttt{application/config.php} seperti berikut:

\begin{lstlisting}[language=PHP, caption=\textit{File application/config.php}, label=kode:querystring]
$config['enable_query_strings'] = FALSE;
$config['controller_trigger'] = 'c';
$config['function_trigger'] = 'm';
\end{lstlisting}

Pengguna dapat mengubah \verb|enable_query_strings| menjadi \textit{TRUE}. Pengubahan fitur tersebut dapat memperbolehkan pengguna untuk menambahkan \textit{query} pada \textit{URL} yang dibentuk. Berikut merupakan contoh pengaksesan \textit{URL} melalui \textit{query strings}:

\begin{center}
\verb|index.php?c=controller&m=method|
\end{center}

Sintaks diatas membentuk URL agar dapat mengakses fungsi ataupun metode dari \textit{controller} menggunakan \textit{query} tanpa harus menggunakan \textit{URL} biasa.

\subsection{\textit{Helpers}}
\textit{Helpers} merupakan fungsi pada \textit{CodeIgniter 3}
yang mempermudah pengguna dalam membangun aplikasi web. Setiap \textit{file} \textit{helpers} terdiri dari banyak fungsi yang membantu sesuai kategori dan tidak ditulis dalam format \textit{Object Oriented}. \textit{File helpers} terdapat pada direktori \texttt{system/helpers} atau \texttt{application/helpers}. Pengguna dapat memakai fitur \textit{helpers} dengan cara memuatnya seperti berikut:

\begin{center}
\verb|$this->load->helper('name');|
\end{center}

Pemanggilan \textit{helper} tidak menggunakan ekteksi \texttt{.php} melainkan hanya menggunakan nama dari \textit{helper} yang ingin digunakan. Pengguna dapat memanggil satu atau banyak \textit{helper} pada metode \textit{controller} ataupun \textit{view} sesudah dimuat.

\subsection{\textit{Libraries}}
\textit{CodeIgniter 3} menyediakan \textit{library} yang dapat dipakai pengguna untuk mempermudah pembentukan aplikasi web. \textit{Library} merupakan kelas yang tersedia pada direktori \texttt{application/libraries} dan dapat ditambahkan, diperluas, dan digantikan. 

\begin{lstlisting}[language=PHP, caption=Contoh kelas \textit{library} pada \textit{CodeIgniter 3}, label=kode:libraryclass]
<?php
defined('BASEPATH') OR exit('No direct script access allowed');

class Someclass {

        public function some_method()
        {
        }
}
\end{lstlisting}

Kode \ref{kode:libraryclass} menunjukan contoh \textit{file library} pada \textit{CodeIgniter 3}. Setiap pembentukan \textit{file library} diperlukan huruf kapital dan harus sama dengan nama kelasnya. Kode \ref{kode:libraryclassinit} menunjukan contoh pemanggilan \textit{file} \textit{library} pada \textit{file} \textit{controller}:

\begin{lstlisting}[language=PHP, caption=Contoh pemanggilan \textit{library} pada \textit{file} \textit{controller}, label=kode:libraryclassinit]
$params = array('type' => 'large', 'color' => 'red');

$this->load->library('someclass', $params);
\end{lstlisting}

Kode \ref{kode:libraryclassinit} menunjukan pemanggilan \textit{library} \texttt{someclass} yang dilakukan melalui \textit{controller} manapun. Kelas \textit{library} juga dapat diberikan parameter sesuai dengan metode yang dibentuk. Selain membentuk \textit{library}, \textit{CodeIgniter 3} menyediakan berbagai \textit{library} yang dapat digunakan oleh pengguna. Berikut merupakan \textit{library} yang disediakan oleh \textit{CodeIgniter 3}:

\subsubsection{Kelas \textit{Input}}
Kelas \textit{input} memiliki dua buah fungsi yakni melakukan praproses data masukan dan memberikan metode kepada beberapa \textit{helper}. Penggunaan kelas \textit{input} dapat dipanggil menggunakan sintaks berikut:
\begin{center}
	\verb|$something = $this->input->post('something');|
\end{center}
Sintaks diatas akan mengambil data yang dikirim menggunakan metode \texttt{post} pada data bernama \texttt{something}.

\subsubsection{\textit{JavaScript}}
Penggunaan kelas \textit{javascript} dapat dipanggil pada konstruktor \textit{controller} dengan melakukan inisiasi \textit{library} seperti pada sintaks berikut:

\begin{center}
\verb|$this->load->library('javascript');|
\end{center}

Pengguna selanjutnya harus melakukan inisiasi \textit{library} pada halaman \textit{view tag} \verb|<head>|. Kode \ref{kode:jshead} menunjukan penambahan \textit{tag} pada halaman \textit{view}.

\begin{lstlisting}[language=PHP, caption=Contoh penambahan \textit{tag} pada \textit{file} \textit{view}, label=kode:jshead]
<?php echo $library_src;?>

<?php echo $script_head;?>
\end{lstlisting}

Sintaks \verb|$library_src| merupakan lokasi \textit{library} yang akan dimuat sedangkan \verb|$script_head| merupakan lokasi untuk fungsi yang akan dijalankan. Selanjutnya \textit{javascript} dapat diinisiasikan pada \textit{controller} menggunakan sintaks berikut:

\begin{center}
	\verb|$this->javascript|
\end{center}

Selain menggunakan \textit{javascript}, pengguna dapat memakai \textit{jQuery} dengan menambahkan \textit{jQuery} pada akhir inisiasi kelas \textit{javascript} seperti berikut:

\begin{center}
\verb|$this->load->library('javascript/jquery');|
\end{center}

Penggunga dapat memakai berbagai fungsi \textit{library jquery} menggunakan sintaks berikut:

\begin{center}
\verb|$this->jquery|
\end{center}

\subsubsection{Kelas \textit{Email}}
\textit{CodeIgniter 3} menyediakan kelas \textit{email} dengan fitur sebagai berikut:

\begin{itemize}
\item Beberapa Protokol: \textit{Mail, Sendmail}, dan SMTP
\item Enkripsi TLS dan SSL untuk SMTP
\item Beberapa Penerima
\item CC dan BCCs
\item HTML atau \textit{email} teks biasa
\item Lampiran
\item Pembungkus kata
\item Prioritas
\item Mode BCC \textit{Batch}, memisahkan daftar \textit{email} besar menjadi beberapa BCC kecil.
\item Alat \textit{Debugging email}
\end{itemize}

Penggunaan \textit{library email} dapat dikonfigurasikan pada \textit{file} \texttt{config}. Pengguna dapat mengirim \textit{email} menggunakan fungsi-fungsi yang telah disediakan \textit{library email}. Kode \ref{kode:emaillib} menunjukan contoh pengiriman email dari \texttt{your@example.com} menuju \texttt{someone@example.com} melalui \textit{controller}.

\begin{lstlisting}[language=PHP, caption=Contoh pengiriman email melalui \textit{controller}, label=kode:emaillib]
$this->load->library('email');

$this->email->from('your@example.com', 'Your Name');
$this->email->to('someone@example.com');
$this->email->cc('another@another-example.com');
$this->email->bcc('them@their-example.com');

$this->email->subject('Email Test');
$this->email->message('Testing the email class.');

$this->email->send();
\end{lstlisting}

Kode \ref{kode:emaillib} juga mengirim dua buah salinan yang akan di CC menuju \texttt{another@another-example.com} dan BCC menuju \texttt{them@their-example.com} dengan subjek berupa \texttt{Email Test} dengan pesan \texttt{Testing the email class}. Selain itu, pengguna juga dapat melakukan konfigurasi preferensi \textit{email} melalui dua puluh satu preferesi. Pengguna dapat melakukan konfigurasi secara otomatis melalui \textit{file config} atau melakukan konfigurasi secara manual. Kode \ref{kode:emaillibset} menunjukan contoh konfigurasi preferensi secara manual untuk mengirimkan \textit{email}.

\begin{lstlisting}[caption=Contoh konfigurasi preferensi \textit{library email} secara manual, label=kode:emaillibset]
$config['protocol'] = 'sendmail';
$config['mailpath'] = '/usr/sbin/sendmail';
$config['charset'] = 'iso-8859-1';
$config['wordwrap'] = TRUE;

$this->email->initialize($config);
\end{lstlisting}
Kode \ref{kode:emaillibset} menunjukan contoh konfigurasi pengiriman \textit{email} dengan protokol \texttt{sendmail}, tujuan \texttt{usr/sbin/sendmail}, \textit{character set} berjenis \texttt{iso-8859-1}, dan menyalakan fitur \texttt{wordwrap}. Selanjutnya konfigurasi dapat diinisialisasikan menggunakan sintaks \texttt{initialize}.

\subsubsection{Kelas \textit{File Uploading}}
Pengunggahan \textit{file} terdapat empat urutan proses sebagai berikut:
\begin{enumerate}
\item Dibentuk sebuah form untuk pengguna memilih dan mengunggah \textit{file}.
\item Setelah \textit{file} diunggah, \textit{file} akan dipindahkan menuju direktori yang dipilih.
\item Pada pengiriman dan pemindahan \textit{file} dilakukan validasi sesuai dengan ketentuan yang ada.
\item Setelah \textit{file} diterima akan dikeluarkan pesan berhasil.
\end{enumerate}

Perangkat lunak akan memindahkan \textit{file} yang sudah diunggah pada \textit{form} menuju \textit{controller} untuk dilakukan validasi dan penyimpanan. \textit{File} yang tidak berhasil dilakukan validasi akan mengembalikan pesan dan tampilan \textit{error}.

\begin{lstlisting}[language=PHP, caption=Contoh \textit{controller} untuk melakukan validasi dan penyimpanan, label=kode:controllerfileupload]
<?php

class Upload extends CI_Controller {

        public function __construct()
        {
                parent::__construct();
                $this->load->helper(array('form', 'url'));
        }

        public function index()
        {
                $this->load->view('upload_form', array('error' => ' ' ));
        }

        public function do_upload()
        {
                $config['upload_path']          = './uploads/';
                $config['allowed_types']        = 'gif|jpg|png';
                $config['max_size']             = 100;
                $config['max_width']            = 1024;
                $config['max_height']           = 768;

                $this->load->library('upload', $config);

                if ( ! $this->upload->do_upload('userfile'))
                {
                        $error = array('error' => $this->upload->display_errors());

                        $this->load->view('upload_form', $error);
                }
                else
                {
                        $data = array('upload_data' => $this->upload->data());

                        $this->load->view('upload_success', $data);
                }
        }
}
?>
\end{lstlisting}

Kode \ref{kode:controllerfileupload} menunjukan contoh kode dengan dua buah metode sebagai berikut:
\begin{enumerate}
\item \verb|index()| yang digunakan untuk mengembalikan \textit{view} bernama \texttt{upload\_form}
\item \verb|do_upload| yang digunakan untuk melakukan validasi berupa tipe, ukuran, lebar, dan panjang maksimal sebuah \textit{file}. Metode ini juga mengembalikan \textit{error} apabila pengunggahan \textit{file} tidak berhasil dan menyimpan \textit{file} pada direktori \texttt{./uploads/}.
\end{enumerate}
Direktori penyimpanan dapat diubah sesuai dengan kebutuhan namun perlu dilakukan penrubahan izin direktori menjadi 777 agar dapat dibaca, ditulis, dan dieksekusi oleh seluruh pengguna.

\subsubsection{\textit{Session}}
\textit{Library session} memperbolehkan aplikasi untuk melihat jejak dari aktivitas pengguna. \textit{CodeIgniter} menyediakan beberapa cara penyimpanan \textit{session} yakni menggunakan \textit{file}, \textit{database}, \textit{redis}, dan \textit{memcached}. \textit{Library} ini dapat diinisiasikan menggunakan sintaks sebagai berikut:
\begin{center}
	\verb|$this->load->library('session');|
\end{center}
Sintaks diatas akan melakukan inisiasi terhadap \textit{library session} dan dapat digunakan menggunakan sintaks sebagai berikut:
\begin{center}
	\verb|$this->session|
\end{center}
\textit{Session} bekerja ketika halaman termuat dan akan melakukan pengecekan terhadap \textit{session cookie} pengguna. Data ini akan disimpan dan diperbaharui sesuai dengan  data yang diterima. Pengguna dapat mengambil data dari \textit{session} ini menggunakan sintaks sebagai berikut:
\begin{center}
	\verb|$this->session->userdata('item');|
\end{center}
Sintaks diatas akan mengambil seluruh data \textit{session} dengan nama \textit{item}. Secara \textit{default}, aplikasi akan menyimpan data \textit{session} pada file namun pengguna dapat mengubah konfigurasi sesuai dengan kebutuhan. Kode \ref{kode:sessiondb} menunjukan konfigurasi untuk menyimpan data pada \textit{database} dengan nama tabel \textit{ci\_sessions}.
\begin{lstlisting}[caption=Contoh konfigurasi untuk menyimpan data pada \textit{database}, label=kode:sessiondb]
$config['sess_driver'] = 'database';
$config['sess_save_path'] = 'ci_sessions';
\end{lstlisting}

\subsubsection{Kelas \textit{URI}}
Kelas \textit{URI} menyediakan metode untuk mengambil informasi yang terdapat pada \textit{URI strings}. Kelas ini diinisiasikan secara otomatis oleh \textit{CodeIgniter} sehingga tidak perlu dilakukan inisiasi lagi. Informasi dapat diambil menggunakan sintaks berikut:
\begin{center}
	\verb|$array = $this->uri->uri_to_assoc(3, $default);|
\end{center}
Sintaks diatas akan mengambil informasi yang tersedia pada \textit{URI segment} menjadi sebuah \textit{associative array}. \textit{Array} yang dihasilkan akan berisikan \textit{key} dan \textit{value} sesuai dengan penempatan \textit{segment}.

\subsubsection{Kelas \textit{Zip Encoding}}
Zip merupakan format sebuah \textit{file} yang berguna untuk melakukan kompress terhadap \textit{file} untuk mengurangi ukuran dan menjadikannya sebuah \textit{file}. \textit{CodeIgniter 3} menyediakan \textit{library} Zip \textit{Encoding} yang dapat digunakan untuk membangun arsip Zip yang dapat diunduh menuju \textit{desktop} atau disimpan pada direktori. \textit{Library} ini dapat diinsiasi dengan kode sebagai berikut:

\begin{center}
\verb|$this->load->library('zip');|
\end{center}

Setelah diinisiasi, pengguna dapat memanggil \textit{library} tersebut menggunakan kode sebagai berikut:

\begin{center}
\verb|$this->zip|
\end{center}

Sintaks diatas akan memanggil \textit{library} Zip pada sebuah \textit{file}. Kode \ref{kode:zip} menunjukan contoh penggunaan \textit{library Zip Encoding} untuk melakukan penyimpanan \textit{Zip file} pada direktori dan dapat mengunduh \textit{file} menuju \textit{desktop} pengguna.

\begin{lstlisting}[language=PHP, caption=Contoh penggunaan \textit{library Zip Encoding}, label=kode:zip]
$name = 'mydata1.txt';
$data = 'A Data String!';

$this->zip->add_data($name, $data);

// Write the zip file to a folder on your server. Name it "my_backup.zip"
$this->zip->archive('/path/to/directory/my_backup.zip');

// Download the file to your desktop. Name it "my_backup.zip"
$this->zip->download('my_backup.zip');
\end{lstlisting}

Selain menggunakan \textit{library} yang sudah disediakan, pengguna dapat membangun dan memperluas \textit{libraries} sendiri sesuai dengan kebutuhan. Kode \ref{kode:newlib} merupakan contoh \textit{library} yang dibentuk oleh pengguna yang berfungsi untuk memanggil \textit{helper} bernama URL.

\begin{lstlisting}[language=PHP, caption=Contoh \textit{library} yang dibentuk, label=kode:newlib]
class Example_library {

        protected $CI;

        // We'll use a constructor, as you can't directly call a function
        // from a property definition.
        public function __construct()
        {
                // Assign the CodeIgniter super-object
                $this->CI =& get_instance();
        }

        public function foo()
        {
                $this->CI->load->helper('url');
                redirect();
        }
}
\end{lstlisting}

Pengguna dapat memanggil \textit{library} tersebut seperti memanggil kelas \textit{library} lainnya. Selain itu, pengguna juga dapat mengganti \textit{library} yang sudah ada dengan \textit{library} yang dibentuk pengguna. \textit{Library} yang dibentuk harus memiliki nama kelas sama persis dengan nama \textit{library} yang ingin digantikan.

\subsection{\textit{Database}}
\textit{CodeIgniter 3} memiliki konfigurasi \textit{database} yang menyimpan data-data terkait aturan \textit{database}. Kode \ref{kode:databaseconf} menunjukan contoh konfigurasi pada file \textit{database} untuk \textit{database} bernama \texttt{database\_name} dengan \textit{username} \texttt{root} tanpa sebuah \textit{password}.

\begin{lstlisting}[caption=Contoh konfigurasi \textit{database}, label=kode:databaseconf]
$db['default'] = array(
        'dsn'   => '',
        'hostname' => 'localhost',
        'username' => 'root',
        'password' => '',
        'database' => 'database_name',
        'dbdriver' => 'mysqli',
        'dbprefix' => '',
        'pconnect' => TRUE,
        'db_debug' => TRUE,
        'cache_on' => FALSE,
        'cachedir' => '',
        'char_set' => 'utf8',
        'dbcollat' => 'utf8_general_ci',
        'swap_pre' => '',
        'encrypt' => FALSE,
        'compress' => FALSE,
        'stricton' => FALSE,
        'failover' => array()
);
\end{lstlisting}

\textit{CodeIgniter 3} juga menyediakan fitur \textit{query} untuk menyimpan, memasukan, memperbarui, dan menghapus data pada \textit{database} sesuai dengan konfigurasi \textit{database} yang sudah diatur. Kode \ref{kode:databaseselect} menunjukan contoh kode untuk melakukan \textit{query} pada tabel \texttt{blogs} yang melakukan \textit{join} dengan tabel \texttt{comments}. Pengguna dapat mengambil hasil dari \textit{query} menjadi \textit{object} atau \textit{array}.

\begin{lstlisting}[language=PHP, caption=Contoh penggunaan \textit{query}, label=kode:databaseselect]
$this->db->select('*');
$this->db->from('blogs');
$this->db->join('comments', 'comments.id = blogs.id');
$query = $this->db->get();
\end{lstlisting}

Selain untuk melakukan \textit{query}, \textit{database} pada \textit{CodeIgniter 3} juga dapat digunakan untuk membangun, menghapus, dan mengubah \textit{database} ataupun menambahkan kolom pada \textit{table}. Penggunaan \textit{database} untuk membangun, menghapus, atau mengubah \textit{database} harus dilakukan inisasi sebagai berikut:
\begin{center}
\verb|$this->load->dbforge()|
\end{center}
Setelah dilakukan inisiasi pengguna dapat membangun \textit{database} menggunakan kelas \texttt{Forge}. Kode \ref{kode:databasecreate} menunjukan contoh untuk membangun \textit{database} dengan nama \texttt{db\_name}.

\begin{lstlisting}[language=PHP, caption=Contoh membangun \textit{database} menggunakan \textit{CodeIgniter 3} , label=kode:databasecreate]
	$this->dbforge->create_database('db_name')
\end{lstlisting}

Selain itu, pengguna juga dapat menambahkan kolom dengan konfigurasinya. Kode \ref{kode:databasefield} menunjukan contoh penambahan kolom sesuai dengan kebutuhan pengguna.

\begin{lstlisting}[language=PHP, caption=Contoh menambahkan kolom dengan konfigurasinya menggunakan \textit{CodeIgniter3} , label=kode:databasefield]
	$fields = array(
        'blog_id' => array(
                'type' => 'INT',
                'constraint' => 5,
                'unsigned' => TRUE,
                'auto_increment' => TRUE
        ),
        'blog_title' => array(
                'type' => 'VARCHAR',
                'constraint' => '100',
                'unique' => TRUE,
        ),
        'blog_author' => array(
                'type' =>'VARCHAR',
                'constraint' => '100',
                'default' => 'King of Town',
        ),
        'blog_description' => array(
                'type' => 'TEXT',
                'null' => TRUE,
        ),
);
$this->dbforge->add_field($fields)
$this->dbforge->create_table('table_name');
\end{lstlisting}
Kode \ref{kode:databasefield} menunjukan contoh penggunaan \textit{database} untuk menambahkan kolom sesuai dengan tipe, batas dari data yang disimpan, penambahan otomatis, dan isi \textit{default} dari kolom pada tabel \texttt{table\_name}.

\subsection{\textit{URI Routing}}
\textit{URL string} yang dihasilkan pada \textit{CodeIgniter 3} biasanya menggunakan nama atau metode \textit{controller} seperti pada berikut:

\begin{center}
\verb|example.com/class/function/id/|
\end{center}

Namun, pengguna dapat melakukan pemetaan ulang terhadap url yang dibentuk agar dapat memanggil metode dengan penambahaan \textit{segment} yang diinginkan.

\begin{lstlisting}[caption=Contoh url yang sudah dimetakan,label=kode:urlrouting]
example.com/product/1/
example.com/product/2/
example.com/product/3/
example.com/product/4/
\end{lstlisting}

Kode \ref{kode:urlrouting} merupakan contoh url yang sudah dimetakan ulang. Pengguna dapat menambahkan kode pemetaan pada \textit{file application/config/routes.php} yang terdapat \textit{array} bernama \verb|$route|. Berikut merupakan beberapa cara melakukan pemetaan terhadap \textit{url}:

\subsubsection{WildCards}

\textit{Route wildcard} biasanya berisikan kode seperti berikut:

\begin{center}
\verb|$route['product/:num'] = 'catalog/product_lookup';|
\end{center}

\textit{Route} diatas dibagi menjadi dua buah yakni:
\begin{enumerate}
\item Bagian segmen \textit{URL}

Bagian pertama merupakan segmen pertama \textit{url} yang akan tampil pada \textit{url}. Bagian kedua merupakan segmen kedua dapat berisikan angka atau karakter.

\item Bagian kelas dan metode

Bagian kedua berisikan kelas dan metode dari \textit{controller} yang akan digunakan pada \textit{url}.

\end{enumerate} 

\subsubsection{Ekspresi Reguler}

Pengguna dapat memakai ekspresi reguler untuk melakukan pemetaan ulang \textit{route}. Berikut merupakan contoh ekspresi reguler yang biasa digunakan:

\begin{center}
\verb|$route['products/([a-z]+)/(\d+)'] = '$1/id_$2';|
\end{center}

Ekpresi diatas menghasilkan URI \texttt{products/shirts/123} yang memanggil kelas \textit{controller} dan metode \texttt{id\_123}. Pengguna juga dapat mengambil segmen banyak seperti berikut:
\begin{center}
\verb|$route['login/(.+)'] = 'auth/login/$1';|
\end{center}

\subsection{\textit{Error Handling}}
\textit{CodeIgniter 3} menyediakan sebuah fungsi untuk melaporkan \textit{error} pada aplikasi. Fungsi ini juga memungkinkan \textit{error} dan \textit{debugging message} disimpan menuju file \textit{text}. Berikut merupakan contoh penggunaan fungsi ini:
\begin{center}
\verb|$show_error($message, $status_code, $heading = 'An Error Was Encountered')|
\end{center}

Sintaks diatas menerima parameter berupa pesan, kode \textit{HTTP}, dan \textit{heading} yang ditampilkan pada halaman \textit{error}. Sintaks ini akan mengembalikan halaman \textit{error} dengan nama \texttt{error\_general.php}.

\subsection{\textit{Auto-loading}}
\textit{CodeIgniter 3} menyediakan sebuah fungsi untuk memuat berbagai kelas seperti \textit{libraries, helpers,} dan \textit{models}. Kelas dapat dimasukan pada \texttt{application/config/autoload.php} sesuai dengan petunjuk yang terdapat pada \textit{file}. \textit{File autoload} akan diinisiasikan setiap aplikasi dijalankan sehingga pengguna tidak perlu memuat kelas tersebut berulang kali. 

\section{SharIF Judge \cite{sharif:23}}
\label{sec:judge}

\textit{SharIF Judge} merupakan sebuah \textit{Online Judge} percabangan dari \textit{Sharif Judge} yang dibentuk oleh Mohammed Javad Naderi. \textit{Sharif Judge} dibentuk menggunakan \textit{CodeIgniter 3} dan dimodifikasi sesuai dengan kebutuhan di Informatika Universitas Katolik Parahyangan menjadi nama \textit{SharIF Judge}. \textit{SharIF Judge} dapat menilai kode berbahasa \textit{C, C++, Java, dan Python} dengan mengunggah \textit{file} ataupun mengetiknya secara langsung.

\subsection{Struktur Aplikasi}
\label{subsec:judgestructure}

Aplikasi \textit{SharIF Judge} menggunakan struktur berbasis MVC yang membagi setiap \textit{file} pada direktorinya. Kode \ref{kode:sharifstructurebab2} menunjukan struktur aplikasi \textit{SharIF Judge}.
\begin{lstlisting}[caption=Struktur aplikasi \textit{SharIF Judge},label=kode:sharifstructurebab2]
|-- application
|   |-- cache
|   |-- config
|   |-- controllers
|   |-- core
|   |-- helpers
|   |-- hooks
|   |-- language
|   |-- libraries
|   |-- logs
|   |-- models
|   |-- third_party
|   |-- vendor
|   |-- views
|-- assets
|   |-- images
|   |-- js
|   |-- styles
|-- restricted
|   |-- tester
|-- system
|-- assignments
\end{lstlisting}

\subsection{Instalasi}
\label{subsec:instalasi}

\textit{SharIF Judge} memerlukan beberapa persyaratan dan langkah dalam melakukan instalasi. Berikut merupakan persyaratan dan langkah-langkah melakukan instalasi \textit{SharIF Judge}:

\subsubsection{Persyaratan}
\label{subsubsec:persyaratan}
SharIF Judge dapat dijalankan pada sistem operasi \textit{Linux} dengan syarat sebagai berikut:
\begin{itemize}
\item Diperlukan \textit{webserver} dengan versi PHP 5.3 atau lebih baru.
\item Pengguna dapat menjalankan PHP pada \textit{command line}. Pada \textit{Ubuntu} diperlukan instalasi paket \texttt{php5-cli}.
\item \textit{MySQL database} dengan ekstensi \textit{Mysqli} untuk PHP atau \textit{PostgreSql database}.
\item PHP harus memiliki akses untuk menjalankan perintah melalui fungsi \textit{shell\textunderscore exec}. Kode \ref{kode:shell} menunjukan sintaks untuk melakukan pengetesan fungsi PHP pada \texttt{shell}.

\begin{lstlisting}[caption=Kode untuk melakukah pengetesan fungsi, label=kode:shell]
	echo shell_exec("php -v");
\end{lstlisting}

\item \textit{Tools} untuk melakukan kompilasi dan menjalankan kode yang dikumpulkan (\textit{gcc, g++, javac, java, python2, python3}).
\item \textit{Perl} disarankan untuk diinstalasi untuk alasan ketepatan waktu, batas memori, dan memaksimalkan batas ukuran pada hasil kode yang dikirim.
\end{itemize}

\subsubsection{Instalasi}
\begin{enumerate}
\item Mengunduh versi terakhir dari \textit{SharIF Judge} dan melakukan \textit{unpack} pada direktori \textit{public html}.
\item Memindahkan \textit{folder system} dan \textit{application} diluar direktori \textit{public} dan mengubah \textit{path} pada \verb|index.php|(Opsional). Kode \ref{kode:movesystemandapp} menunjukan contoh path pada \textit{file} \texttt{index.php}.

\begin{lstlisting}[language=PHP, caption=Contoh \textit{path} pada halaman index.php, label=kode:movesystemandapp]
	$system_path = '/home/mohammad/secret/system';
	application_folder = '/home/mohammad/secret/application';
	
\end{lstlisting}
\item Membangun \textit{database MySql} atau \textit{PostgreSql} untuk \textit{SharIF Judge}. Jangan melakukan instalasi paket koneksi \textit{database} apapun untuk \textit{C, C++, Java,} atau \textit{Python}.
\item Mengatur koneksi \textit{database} pada \textit{file} \verb|application/config/database.example.php| dan menyimpannya dengan nama \verb|database.php|. Pengguna dapat menggunakan awalan untuk nama tabel. Kode \ref{kode:dbsetup} menunjukan contoh pengaturan koneksi untuk nama, \textit{host}, dan prefix pada \textit{database}.

\begin{lstlisting}[caption=Contoh pengaturan koneksi untuk \textit{database}, label=kode:dbsetup]
/*  Enter database connection settings here:  */
'dbdriver' => 'postgre',    // database driver (mysqli, postgre)
'hostname' => 'localhost',  // database host
'username' => `,           // database username
'password' => `,           // database password
'database' => `,           // database name
'dbprefix' => 'shj_',       // table prefix
\end{lstlisting}
\item Mengatur \textit{RADIUS} \textit{server} dan \textit{mail server} pada \textit{file} \verb|application/config/secrets.example.php| dan menyimpannya dengan nama \verb|secrets.php|.
\item Mengatur \verb|application/cache/Twig| agar dapat ditulis oleh PHP.
\item Membuka halaman utama SharIF Judge pada \textit{web browser} dan mengikuti proses instalasi.
\item Melakukan \textit{Log in} dengan akun admin.
\item Memindahkan direktori \verb|tester| dan \verb|assignments| diluar direktori publik dan mengatur kedua direktori agar dapat ditulis oleh PHP. Selanjutnya Menyimpan \textit{path} kedua direktori pada halaman \textit{Settings}. Direktori \verb|assignments| digunakan untuk menyimpan \textit{file-file} yang diunggah agar tidak dapat diakses publik.

\end{enumerate}

\subsection{\textit{Clean URLs}}
\label{sec:cleanurls}
Secara \textit{default}, \verb|index.php| merupakan bagian dari seluruh \textit{urls} pada SharIF judge. Kode \ref{kode:judgecleanursl1} menunjukan contoh dari URL SharIF Judge.

\begin{lstlisting}[language=PHP, caption=Contoh URL \textit{SharIF Judge}, label=kode:judgecleanursl1]
\verb|http://example.mjnaderi.ir/index.php/dashboard|

\verb|http://example.mjnaderi.ir/index.php/users/add|
\end{lstlisting}

Pengguna dapat menghapus \verb|index.php| pada URL dan mendapatkan URL yang baik apabila sistem pengguna mendukung \textit{URL rewriting}. Kode \ref{kode:judgecleanursl2} menunjukan hasil URL yang telah ditulis ulang.

\begin{lstlisting}[language=PHP, caption=Contoh hasil URL \textit{SharIF Judge} yang telah ditulis ulang, label=kode:judgecleanursl2]
\verb|http://example.mjnaderi.ir/dashboard|

\verb|http://example.mjnaderi.ir/users/add|
\end{lstlisting}

\subsubsection{Cara Mengaktifkan \textit{Clean URLs}}
Pengguna dapat mengaktifkan \textit{clean} url dengan cara sebagai berikut:
\begin{itemize}
\item Mengganti nama \textit{file} \verb|.htaccess2| pada direktori utama menjadi \verb|.htaccess|.
\item Mengganti \verb|$config['index_page'] = 'index.php';| menjadi \verb|$config['index_page'] = '';| pada \textit{file} \verb|application\config\config.php|.
\end{itemize}

\subsection{Users}
\label{sec:users}
Pada perangkat lunak SharIF Judge, pengguna dibagi menjadi 4 buah. Keempat pengguna tersebut adalah \textit{Admins, Head Instructors, Instructors}, dan\textit{ Students}. Tabel \ref{tab:userlevel} menunjukan pembagian tingkat setiap pengguna. Setiap pengguna memiliki akses untuk aksi yang berbeda berdasarkan tingkatnya. Tabel \ref{tab:userPermission} merupakan aksi yang dapat dilakukan oleh setiap pengguna.

\begin{table}[H]
	\centering 
	\caption{\textit{Tabel tingkat pengguna}}
	\label{tab:userlevel}
	\begin{tabular}{|c|c|}
		\hline
		\textit{\textbf{User Role}} & \textit{\textbf{User Level}} \\ \hline
		\textit{Admin} & 3 \\ \hline
		\textit{Head Instructor} & 2 \\ \hline
		\textit{Instructor} & 1 \\ \hline
		\textit{Student} & 0 \\ \hline		
	\end{tabular} 
\end{table}

\begin{table}[H] 
	\centering 
	\caption{\textit{Tabel izin aksi setiap pengguna}}
	\label{tab:userPermission}
	\begin{tabular}{|l|c|c|c|c|}
		\hline
		Aksi & \textit{Admin} & \textit{Head Instructor} & \textit{Instructor} & \textit{Student} \\
		
		\hline
		Mengubah \textit{Settings} & \ding{51} & \ding{53} & \ding{53} & \ding{53} \\
		Menambah/Menghapus Pengguna & \ding{51} & \ding{53} & \ding{53} & \ding{53} \\
		Mengubah Peran Pengguna & \ding{51} & \ding{53} & \ding{53} & \ding{53} \\
		Menambah/Menghapus/Mengubah \textit{Assignment} & \ding{51} & \ding{51} & \ding{53} & \ding{53} \\
		Mengunduh \textit{Test} & \ding{51} & \ding{51} & \ding{53} & \ding{53} \\
		
		Menambah/Menghapus/Mengubah Notifikasi & \ding{51} & \ding{51} & \ding{53} & \ding{53} \\
		\textit{Rejudge} & \ding{51} & \ding{51} & \ding{53} & \ding{53} \\
		Melihat/\textit{Pause}/Melanjutkan/\textit{Submission Queue} & \ding{51} & \ding{51} & \ding{53} & \ding{53} \\
		Mendeteksi Kode yang Mirip & \ding{51} & \ding{51} & \ding{53} & \ding{53} \\
		Melihat Semua Kode & \ding{51} & \ding{51} & \ding{51} & \ding{53} \\
		
		Mengunduh Kode Final& \ding{51} & \ding{51} & \ding{51} & \ding{53} \\
		Memilih \textit{Assignment} & \ding{51} & \ding{51} & \ding{51} & \ding{51} \\
		\textit{Submit} & \ding{51} & \ding{51} & \ding{51} & \ding{51} \\
		
		\hline
		
	\end{tabular} 
\end{table}

\subsubsection{Menambahkan Pengguna}

Admin dapat menambahkan pengguna melalui bagian \textit{Add User} pada halaman \textit{Users}. Admin harus mengisi setiap informasi dimana baris yang diawali \# merupakan komen dan setiap baris lainnya mewakili pengguna. Kode \ref{kode:addUser} menunjukan sintaks untuk menambah pengguna.
\begin{lstlisting}[caption=Sintaks untuk menambahkan pengguna, label=kode:addUser]
USERNAME,EMAIL,DISPLAY-NAME,PASSWORD,ROLE

* Usernames may contain lowercase letters or numbers and must be between 3 and 20 characters in length.
* Passwords must be between 6 and 30 characters in length.
* You can use RANDOM[n] for password to generate random n-digit password.
* ROLE must be one of these: `admin`, `head_instructor`, `instructor`, `student`
\end{lstlisting}

Kode \ref{kode:addUserExample} menunjukan contoh untuk menambahkan pengguna.
\begin{lstlisting}[caption=Contoh kode untuk menambahkan pengguna, label=kode:addUserExample]
# This is a comment!
# This is another comment!
instructor,instructor@sharifjudge.ir,Instructor One,123456,head_instructor
instructor2,instructor2@sharifjudge.ir,Instructor Two,random[7],instructor
student1,st1@sharifjudge.ir,Student One,random[6],student
student2,st2@sharifjudge.ir,Student Two,random[6],student
student3,st3@sharifjudge.ir,Student Three,random[6],student
student4,st4@sharifjudge.ir,Student Four,random[6],student
student5,st5@sharifjudge.ir,Student Five,random[6],student
student6,st6@sharifjudge.ir,Student Six,random[6],student
student7,st7@sharifjudge.ir,Student Seven,random[6],student
\end{lstlisting}

\subsection{Menambah \textit{Assignment}}
\label{sec:addAssignment}
Pengguna dapat menambahkan \textit{assignment} baru melalui bagian \textit{Add} pada halaman \textit{Assignments} yang ditunjukan pada Gambar \ref{fig:addassignment}.
\begin{figure}[H]
	\centering  
	\includegraphics[scale=0.3]{addassignment}  
	\caption[Tampilan halaman \textit{SharIF Judge} untuk menambahkan \textit{assignment}]{Tampilan halaman \textit{SharIF Judge} untuk menambahkan \textit{assignment}} 
	\label{fig:addassignment} 
\end{figure} 

Berikut merupakan beberapa pengaturan pada halaman \textit{Add Assignments}:
\begin{itemize}
\item \subsubsection{\textit{Assignment Name}}
\textit{Assignment} akan ditampilkan sesuai dengan masukan pada daftar \textit{assignment}.

\item \subsubsection{\textit{Start Time}}
Pengguna tidak dapat mengumpulkan \textit{assignment} sebelum waktu dimulai("\texttt{Start Time}"). Format pengaturan waktu untuk waktu mulai adalah \verb|MM/DD/YYYY HH:MM:SS| dengan contoh \verb|08/31/2013 12:00:00|. 

\item \subsubsection{\textit{Finish Time, Extra Time}}
Pengguna tidak dapat mengumpulkan \textit{assignment} setelah \textit{Finish Time} dan \textit{Extra Time} berakhir. Pengumpulan \textit{Assignment} pada \textit{Extra Time} akan dikalikan sesuai dengan koefisien. Pengguna harus menulis skrip PHP untuk menghitung koefisien pada \textit{field Coefficient Rule}. Format pengaturan waktu untuk waktu selesai sama seperti waktu mulai yakni \verb|MM/DD/YYYY HH:MM:SS| dan format waktu tambahan menggunakan menit dengan contoh \verb|120| (2 jam) atau \verb|48*60| (2 hari).  

\item \subsubsection{\textit{Participants}}
Pengguna dapat memasukan \textit{username} setiap partisipan atau menggunakan kata kunci \textit{ALL} untuk membiarkan seluruh pengguna melakukan pengumpulan. Contoh: \texttt{admin, instructor1, instructor2 ,student1 , student2 , student3, student4}.

\item \subsubsection{\textit{Tests}}
Pengguna dapat mengunggah \textit{test case} dalam bentuk \textit{zip file} sesuai dengan struktur pada subbab \ref{subsec:testStructure}.

\item \subsubsection{\textit{Open}}
Pengguna dapat membuka dan menutup \textit{assigment} untuk pengguna \textit{student} melalui pilihan ini. Pengguna selain \textit{student} tetap dapat mengumpukan \textit{assignment} apabila \textit{assignment} sudah ditutup.

\item \subsubsection{\textit{Score Board}}
Pengguna dapat menghidupkan dan mematikan \textit{score board} melalui pilihan ini.

\item \subsubsection{\textit{Java Exceptions}}
Pengguna dapat menghidupkan dan mematikan fungsi untuk menunjukan \textit{java exceptions} kepada pengguna \textit{students} dan tidak akan memengaruhi kode yang sudah di \textit{judge} sebelumnya. Kode \ref{kode:javaexceptions} menunjukan tampilan apabila fitur \textit{java exceptions} dinyalakan:
\begin{lstlisting}[language=Java, caption=Contoh tampilan fitur \textit{Java Exceptions}, label=kode:javaexceptions]
Test 1
ACCEPT
Test 2
Runtime Error (java.lang.ArrayIndexOutOfBoundsException)
Test 3
Runtime Error (java.lang.ArrayIndexOutOfBoundsException)
Test 4
ACCEPT
Test 5
ACCEPT
Test 6
ACCEPT
Test 7
ACCEPT
Test 8
Runtime Error (java.lang.ArrayIndexOutOfBoundsException)
Test 9
Runtime Error (java.lang.StackOverflowError)
Test 10
Runtime Error (java.lang.ArrayIndexOutOfBoundsException)
\end{lstlisting}

\item \subsubsection{\textit{Archived Assignment}}
Pengguna dapat menghidupkan fitur ini dan \textit{assignment} akan dibentuk dengan waktu selesai \verb|2038-01-18 00:00:00| (UTC + 7) dengan kata lain pengguna memiliki waktut tidak terhingga untuk mengumpulkan \textit{assignment}.

\item \subsubsection{\textit{Coefficient Rule}}
Pengguna dapat menuliskan skrip PHP pada bagian ini yang akan menghitung koefisien dikalikan dengan skor. Pengguna harus memasukan koefisien dari 100 dalam variabel \verb|$coefficient|. Pengguna dapat menggunakan variabel \verb|$extra_time| dan \verb|$delay|. \verb|$extra_time| merupakan total dari waktu tambahan yang diberikan kepada pengguna dalam detik sedangkan \verb|$delay| merupakan waktu dalam detik yang melewati waktu selesai yang dapat diisikan berupa negatif. Skrip PHP pada bagian ini tidak boleh mengandung tag \verb|<?php|, \verb|<?|, dan \verb|?>|. Kode \ref{kode:coefficientrule} menunjukan contoh skrip dimana \verb|$extra_time| adalah 172800 detik atau 2 hari.

\begin{lstlisting}[language=PHP, caption=Contoh skrip PHP, label=kode:coefficientrule]
if ($delay<=0)
  // no delay
  $coefficient = 100;

elseif ($delay<=3600)
  // delay less than 1 hour
  $coefficient = ceil(100-((30*$delay)/3600));

elseif ($delay<=86400)
  // delay more than 1 hour and less than 1 day
  $coefficient = 70;

elseif (($delay-86400)<=3600)
  // delay less than 1 hour in second day
  $coefficient = ceil(70-((20*($delay-86400))/3600));

elseif (($delay-86400)<=86400)
  // delay more than 1 hour in second day
  $coefficient = 50;

elseif ($delay > $extra_time)
  // too late
  $coefficient = 0;
\end{lstlisting}

\item \subsubsection{\textit{Name}}
Merupakan nama dari masalah pada \textit{assignments}.

\item \subsubsection{\textit{Score}}
Merupakan skor dari masalah pada \textit{assignments}. 

\item \subsubsection{\textit{Time Limit}}
Pengguna dapat menentukan batas waktu untuk menjalankan kode dalam satuan milidetik. Bahasa \textit{Python} dan \textit{Java} biasanya memiliki waktu lebih lambat dari \textit{C/C++} sehingga membutuhkan waktu lebih lama.

\item \subsubsection{\textit{Memory Limit}}
Pengguna dapat menentukan batas memori dalam \textit{kilobytes}, namun penggunan pembatasan memori tidak terlalu akurat.

\item \subsubsection{\textit{Allowed Languages}}
Pengguna dapat menentukan bahasa setiap masalah pada \textit{assignment} yang dipisahkan oleh koma. Terdapat beberapa bahasa yang tersedia yaitu \textit{C, C++, Java, Python 2, Python 3, Zip, PDF}, dan \textit{TXT}. Pengguna dapat memakai \textit{Zip, PDF}, dan \textit{TXT} apabila opsi \textit{Upload Only} dinyalakan. Contoh : \texttt{C, C++, Zip} atau \texttt{Python 2,Python 3}.

\item \subsubsection{\textit{Diff Command}}
\textit{Diff Command} digunakan untuk membandingkan hasil keluaran dengan keluaran yang benar. Secara \textit{default}, \textit{SharIF Judge} menggunakan \verb|diff|, namun pengguna dapat menggantinya pada bagian ini dan bagian ini tidak boleh mengandung spasi.

\item \subsubsection{\textit{Diff Arguments}}
Pengguna dapat mengatur argumen untuk \textit{diff arguments} pada bagian ini. Pengguna dapat melihat \verb|man diff| untuk daftar lengkap argumen \textit{diff}. \textit{SharIF Judge} terdapat dua buah opsi baru yakni \verb|ignore| dan \verb|identical|.
\begin{itemize}
\item \verb|ignore| : \textit{SharIF Judge} mengabaikan semua baris baru dan spasi.
\item \verb|identical| : \textit{SharIF Judge} tidak mengabaikan apapun namun, keluaran dari \textit{file} yang dikumpulkan harus identik dengan \textit{test case} agar dapat diterima. 
\end{itemize}

\item \subsubsection{\textit{Upload Only}}
Pengguna dapat menghidupkan \textit{Upload only}, namun \textit{SharIF Judge} tidak akan menilai masalah tersebut. Pengguna dapat memakai Zip, \textit{PDF}, dan \textit{TXT} pada \textit{allowed languanges} apabila pengguna menghidupkan bagian ini.
\end{itemize}

\subsection{\textit{Sample Assignment}}
Berikut merupakan contoh dari \textit{assignment} untuk melakukan pengujian perangkat lunak \textit{SharIF Judge}. Penambahan \textit{Assignment} dapat dilakukan dengan memencet tombol \textit{Add} pada halaman \textit{Assignment}.

\subsubsection{\textit{Problems}}
\begin{enumerate}
\item \textit{Problem 1 (Sum)}:
Program pengguna dapat membaca \textit{integer n}, membaca \textit{n integers} dan mengeluarkan hasil dari \textit{integer} tersebut. Tabel \ref{tab:problem1} menunjukan contoh pmasukan dan keluaran untuk \textit{Problem} 1.

\begin{table}[H]
\centering 
\caption{\textit{Tabel contoh masukan dan keluaran \textit{Problem} 1}}
\label{tab:problem1}
\begin{tabular}{|l|c|}
\hline
\multicolumn{1}{|c|}{\textit{Sample Input}}             & \textit{Sample Output} \\ \hline
\begin{tabular}[c]{@{}l@{}}5\\ 53 78 0 4 9\end{tabular} & 145                    \\ \hline
\end{tabular}
\end{table}

\item \textit{Problem 2 (Max)}:
Program pengguna dapat membaca \textit{integer n}, membaca \textit{n integer}, dan mengeluarkan total dari dua buah \textit{integer} terbesar diantara \textit{n integer}.
Tabel \ref{tab:problem2} menunjukan contoh pmasukan dan keluaran untuk \textit{Problem} 2.
\begin{table}[H]
\centering 
\caption{Tabel contoh masukan dan keluaran \textit{Problem} 2}
\label{tab:problem2}
\begin{tabular}{|l|c|}
\hline
\multicolumn{1}{|c|}{\textit{Sample Input}}                             & \textit{Sample Output} \\ \hline
\begin{tabular}[c]{@{}l@{}}7\\ 162 173 159 164 181 158 175\end{tabular} & 356                    \\ \hline
\end{tabular}
\end{table}

\item \textit{Problem 3 (Upload)}:
Pengguna dapat mengunggah \textit{file c} dan \textit{zip} dan \textit{problem} ini tidak akan dinilai karena hanya bersifat \textit{Upload Only}.

\end{enumerate}

\subsubsection{Tests}
Pengguna dapat menemukan \textit{file zip} pada direktori \textit{Assignments}. Kode \ref{kode:teststructure} merupakan susunan pohon dari ketiga \textit{problems} diatas.

\begin{lstlisting}[caption=Susunan pohon untuk ketiga \textit{problems}, label=kode:teststructure]
.
|-- p1
|   |-- in
|   |   |-- input1.txt
|   |   |-- input2.txt
|   |   |-- input3.txt
|   |   |-- input4.txt
|   |   |-- input5.txt
|   |   |-- input6.txt
|   |   |-- input7.txt
|   |   |-- input8.txt
|   |   |-- input9.txt
|   |   --- input10.txt
|   |-- out
|   |   --- output1.txt
|   |-- tester.cpp
|   --- desc.md
|-- p2
|   |-- in
|   |   |-- input1.txt
|   |   |-- input2.txt
|   |   |-- input3.txt
|   |   |-- input4.txt
|   |   |-- input5.txt
|   |   |-- input6.txt
|   |   |-- input7.txt
|   |   |-- input8.txt
|   |   |-- input9.txt
|   |   --- input10.txt
|   |-- out
|   |   |-- output1.txt
|   |   |-- output2.txt
|   |   |-- output3.txt
|   |   |-- output4.txt
|   |   |-- output5.txt
|   |   |-- output6.txt
|   |   |-- output7.txt
|   |   |-- output8.txt
|   |   |-- output9.txt
|   |   --- output10.txt
|   |-- desc.md
|   --- Problem2.pdf
|-- p3
|   --- desc.md
--- SampleAssignment.pdf
\end{lstlisting}

\textit{Problem 1} menggunakan metode \textit{"Tester"} untuk mengecek hasil keluaran, sehingga memiliki \textit{file} \verb|tester.cpp|(\textit{Tester Script}). \textit{Problem 2} mengguanakan metode \textit{"Output Comparison"} untuk mengecek hasil keluaran, sehingga memiliki dua buah direktori \textit{in} dan \textit{out} yang berisi \textit{test case}. \textit{Problem 3} merupakan \textit{problem "Upload-Only"} sehingga tidak perlu dilakukan pengecekan.

\subsubsection{\textit{Sample Solutions}}
Berikut merupakan \textit{sample solutions} untuk \textit{problem} 1:

Kode ref{kode:samplesolutions1} menunjukan contoh solusi untuk bahasa \textit{C}.
\begin{lstlisting}[language=C, caption=Contoh skrip PHP, label=kode:samplesolutions1]
#include<stdio.h>
int main(){
	int n;
	scanf("%d",&n);
	int i;
	int sum =0 ;
	int k;
	for(i=0 ; i<n ; i++){
		scanf("%d",&k);
		sum+=k;
	}
	printf("%d\n",sum);
	return 0;
}
\end{lstlisting}

Kode ref{kode:samplesolutions2} menunjukan contoh solusi untuk bahasa \textit{C++}.
\begin{lstlisting}[language=C++, caption=Contoh solusi \textit{problem} 1 bahasa \textit{C++}, label=kode:samplesolutions2]
#include<stdio.h>
int main(){
	int n;
	scanf("%d",&n);
	int i;
	int sum =0 ;
	int k;
	for(i=0 ; i<n ; i++){
		scanf("%d",&k);
		sum+=k;
	}
	printf("%d\n",sum);
	return 0;
}
\end{lstlisting}

Kode ref{kode:samplesolutions3} menunjukan contoh solusi untuk bahasa \textit{Java}.
\begin{lstlisting}[language=Java,  caption=Contoh solusi \textit{problem} 1 bahasa \textit{Java} label=kode:samplesolutions3]
import java.util.Scanner;
class sum
{
	public static void main(String[] args)
	{ 
		Scanner sc = new Scanner(System.in);
		int n = sc.nextInt();
		int sum =0;
		for (int i=0 ; i<n ; i++)
		{
			int a = sc.nextInt();
			sum += a;
		}
		System.out.println(sum); 
	}
}
\end{lstlisting}

Berikut merupakan contoh solusi untuk \textit{problem 2}:

Kode ref{kode:samplesolutions21} menunjukan contoh solusi untuk bahasa \textit{C}
\begin{lstlisting}[language=C, caption=Contoh solusi \textit{problem} 2 bahasa \textit{C}, label=kode:samplesolutions21]
#include<stdio.h>
int main(){
	int n , m1=0, m2=0;
	scanf("%d",&n);
	for(;n--;){
		int k;
		scanf("%d",&k);
		if(k>=m1){
			m2=m1;
			m1=k;
		}
		else if(k>m2)
			m2=k;
	}
	printf("%d",m1+m2);
	return 0;
}
\end{lstlisting}

Kode ref{kode:samplesolutions22} menunjukan contoh solusi untuk bahasa \textit{C++}
\begin{lstlisting}[language=C++, caption=Contoh solusi \textit{problem} 2 bahasa \textit{C++}, label=kode:samplesolutions22]
#include<iostream>
using namespace std;
int main(){
	int n , m1=0, m2=0;
	cin >> n;
	for(;n--;){
		int k;
		cin >> k;
		if(k>=m1){
			m2=m1;
			m1=k;
		}
		else if(k>m2)
			m2=k;
	}
	cout << (m1+m2) << endl ;
	return 0;
}
\end{lstlisting}

\subsection{\textit{Test Structure}}
\label{subsec:testStructure}
Penambahan \textit{assignment} harus disertakan dengan \textit{file zip} berisikan \textit{test cases}. \textit{File zip} ini sebaiknya berisikan \textit{folder} untuk setiap masalah dengan nama \textit{p1, p1}, dan \textit{p3} selain masalah \textit{Upload-Only}.

\subsubsection{Metode Pemeriksaan}
Terdapat dua buah metode untuk melakukan pemeriksaan yakni metode \textit{Input Output} dan metode \textit{Tester}. Berikut merupakan penjelasan untuk setiap metode pemeriksaan:

\begin{enumerate}
\item Metode perbandingan \textit{Input Output}

Pada metode ini, pengguna harus memberi masukan dan keluaran pada \textit{folder problem}. Perangkat lunak akan memberikan \textit{file test input} ke kode pengguna dan melakukan perbandingan dengan hasil keluaran kode pengguna. \textit{File input} harus berada didalam \textit{folder in} dengan nama \textit{input1.txt, input2.txt}, dst. \textit{File output} harus berada di dalam \textit{folder out} dengan nama \textit{output1.txt} dan \textit{output2.txt}.

\item Metode perbandingan \textit{Tester}

Pada metode ini, pengguna harus menyediakan \textit{file input test,} sebuah \textit{file C++,} dan \textit{file output test} yang bersifat opsional. Perangkat lunak akan memberikan \textit{file input test} ke kode pengguna dan mengambil keluaran pengguna. Selanjutnya, \verb|tester.cpp| akan mengambil masukan pengguna, keluaran tes, dan keluaran program pengguna. Jika keluaran pengguna benar maka perangkat lunak akan mengembalikan 1 sedangkan apabila salah maka perangkat lunak akan mengembalikan 0. Kode \ref{kode:cpptemplate} menunjukan templat yang dapat digunakan pengguna untuk menuliskan \textit{file} \texttt{tester.cpp}.

\begin{lstlisting}[language=C++,caption=Templat kode \texttt{tester.cpp}, label=kode:cpptemplate]
/*
 * tester.cpp
 */
 
#include <iostream>
#include <fstream>
#include <string>
using namespace std;
int main(int argc, char const *argv[])
{
 
	ifstream test_in(argv[1]);    /* This stream reads from test's input file   */
	ifstream test_out(argv[2]);   /* This stream reads from test's output file  */
	ifstream user_out(argv[3]);   /* This stream reads from user's output file  */
 
	/* Your code here */
	/* If user's output is correct, return 0, otherwise return 1       */
 
	...
 
}
\end{lstlisting}

\subsubsection{\textit{Sample File}}
Pengguna dapat menemukan \textit{file sample test} pada direktori \textit{assignments}. Kode \ref{kode:samplefile} menunjukan contoh dari pohon \textit{file sample test}.

\begin{lstlisting}[caption=Contoh pohon \textit{file} dari \textit{sample test}, label=kode:samplefile]
.
|-- p1
|   |-- in
|   |   |-- input1.txt
|   |   |-- input2.txt
|   |   |-- input3.txt
|   |   |-- input4.txt
|   |   |-- input5.txt
|   |   |-- input6.txt
|   |   |-- input7.txt
|   |   |-- input8.txt
|   |   |-- input9.txt
|   |   --- input10.txt
|   |-- out
|   |   --- output1.txt
|   |-- tester.cpp
|-- p2
|   |-- in
|   |   |-- input1.txt
|   |   |-- input2.txt
|   |   |-- input3.txt
|   |   |-- input4.txt
|   |   |-- input5.txt
|   |   |-- input6.txt
|   |   |-- input7.txt
|   |   |-- input8.txt
|   |   |-- input9.txt
|   |   --- input10.txt
|   |-- out
|   |   |-- output1.txt
|   |   |-- output2.txt
|   |   |-- output3.txt
|   |   |-- output4.txt
|   |   |-- output5.txt
|   |   |-- output6.txt
|   |   |-- output7.txt
|   |   |-- output8.txt
|   |   |-- output9.txt
|   |   --- output10.txt
\end{lstlisting}

\textit{Problem} 1 menggunakan metode perbandingan \textit{Tester}, sehingga memiliki \textit{file} \texttt{tester.cpp}. Kode \ref{kode:cppfileproblem1} menunjukan \textit{file} untuk pengujian \textit{problem} 1.

\begin{lstlisting}[language=C++,caption=Kode metode perbandingan \textit{tester} dengan bahasa \textit{tester.cpp}, label=kode:cppfileproblem1]
/*
 * tester.cpp
 */
 
#include <iostream>
#include <fstream>
#include <string>
using namespace std;
int main(int argc, char const *argv[])
{
 
	ifstream test_in(argv[1]);    /* This stream reads from test's input file   */
	ifstream test_out(argv[2]);   /* This stream reads from test's output file  */
	ifstream user_out(argv[3]);   /* This stream reads from user's output file  */
 
	/* Your code here */
	/* If user's output is correct, return 0, otherwise return 1       */
	/* e.g.: Here the problem is: read n numbers and print their sum:  */
 
	int sum, user_output;
	user_out >> user_output;
 
	if ( test_out.good() ) // if test's output file exists
	{
		test_out >> sum;
	}
	else
	{
		int n, a;
		sum=0;
		test_in >> n;
		for (int i=0 ; i<n ; i++){
			test_in >> a;
			sum += a;
		}
	}
 
	if (sum == user_output)
		return 0;
	else
		return 1;
 
}
\end{lstlisting}

\textit{Problem} 2 menggunakan metode perbandingan \textit{Input Output} sehingga memiliki \textit{folder in} dan \textit{folder out} berisikan \textit{test case}. Sedangkan \textit{problem} 3 merupakan \textit{Upload Only}, sehingga tidak memiliki \textit{folder} apapun.

\end{enumerate}

\subsection{Deteksi Kecurangan}
\textit{SharIF Judge} menggunakan Moss untuk mendeteksi kode yang serupa. Moss atau \textit{Measure of Software Similarity} merupakan sistem otomatis untuk menentukan kesamaan atau kemiripan dalam program. Penggunaan utama Moss adalah untuk melakukan pemeriksaan plagiarisme pada kelas \textit{programming}. Pengguna dapat mengirim hasil kode terakhir atau \textit{Final Submission} menuju \textit{server} Moss dengan satu klik. Penggunaan \textit{Moss} memiliki beberapa hal yang harus diatur oleh pengguna sebagai berikut:

\begin{enumerate}
\item Pengguna harus mendapatkan \textit{Moss User id} dan melakukan pengaturan pada \textit{SharIF Judge}. Untuk mendapatkan \textit{Moss User id}, pengguna dapat mendaftar pada halaman \url{http://theory.stanford.edu/~aiken/moss/}. Pengguna selanjutkan akan mendapatkan \textit{email} berupa skrip \textit{perl} berisikan \textit{user id} pengguna. Kode \ref{kode:perlexample} menunjukan contoh dari potongan skrip \textit{perl}.

\begin{lstlisting}[caption=Contoh potongan skrip \textit{perl}, label=kode:perlexample]

...

$server = 'moss.stanford.edu';
$port = '7690';
$noreq = "Request not sent.";
$usage = "usage: moss [-x] [-l language] [-d] [-b basefile1] ... [-b basefilen] [-m #] [-c \"string\"] file1 file2 file3 ...";

#
# The userid is used to authenticate your queries to the server; don't change it!
#
$userid=YOUR_MOSS_USER_ID;

#
# Process the command line options.  This is done in a non-standard
# way to allow multiple -b's.
#
$opt_l = "c";   # default language is c
$opt_m = 10;
$opt_d = 0;

...

\end{lstlisting}
\texttt{User id} pada skrip \textit{perl} diatas dapat digunakan pada \textit{SharIF Judge} untuk mendeteksi kecurangan. Pengguna dapat menyimpan \texttt{userid} pada halaman \textit{SharIF Judge Moss} dan \textit{SharIF Judge} akan menggunakan \texttt{userid} tersebut untuk mendeteksi kecurangan.

\item \textit{Server} pengguna harus memiliki instalasi \verb|perl| untuk menggunakan Moss.
\item Pengguna dianjurkan untuk mendeteksi kode yang mirip setelah \textit{assignment} selesai karena \textit{SharIF Judge} akan mengirimkan hasil akhir kepada Moss dan pengguna \textit{students} dapat mengganti hasil akhir sebelum \textit{assignment} selesai.
\end{enumerate}

\subsection{Keamanan}
\textit{SharIF Judge} terdapat beberapa langkah untuk melakukan pengaturan keamanan perangkat lunak. Berikut merupakan langkah untuk melakukan pengaturan keamanan \textit{SharIF Judge}:
\begin{enumerate}
\item \subsubsection{Menggunakan \textit{Sandbox}}
Pengguna harus memastikan untuk menggunakan \textit{sandbox} untuk bahasa \textit{C/C++} dan menyalakan \textit{Java Security Manager (Java Policy)} untuk bahasa \textit{java}. Penggunaan \textit{sandbox} dapat dilihat pada subbab \ref{subsec:sandboxing}.

\item \subsubsection{Menggunakan \textit{Shield}}
Pengguna harus memastikan untuk menggunakan \textit{shield} untuk bahasa \textit{C, C++,} dan \textit{Python}. Penggunaan \textit{shield} dapat dilihat pada subbab \ref{subsec:shield}.

\item \subsubsection{Menjalankan sebagai \textit{Non-Priviledge User}}
Pengguna diwajibkan menjalankan kode yang telah dikumpulkan sebagai \textit{non-priviledge user}. \textit{Non-Priviledge User} adalah \textit{user} yang tidak memiliki akses jaringan, tidak dapat menulis \textit{file} apapun, dan tidak dapat membangun banyak proses.

Diasumsikan bahwa PHP dijalankan sebagai pengguna \verb|www-data| pada server. Membangun \textit{user} baru \verb|restricted-user| dan melakukan pengaturan \textit{password}. Selanjutnya, jalankan \verb|sudo visudo| dan tambahkan kode \verb|www-data ALL=(restricted_user) NOPASSWD: ALL| pada baris terakhir \textit{file sudoers}.
\begin{itemize}

\item Pada \textit{file} \verb|tester\runcode.sh| ubah kode:

\begin{lstlisting}[language=C,caption=Kode \textit{runcode.sh} sebelum ditambahkan, label=kode:runcodebefore]
if $TIMEOUT_EXISTS; then
	timeout -s9 $((TIMELIMITINT*2)) $CMD <$IN >out 2>err
else
	$CMD <$IN >out 2>err        
fi
\end{lstlisting}

menjadi Kode \ref{kode:runcodeafter}.
\begin{lstlisting}[language=C,caption=Kode \textit{runcode.sh} setelah ditambahkan, label=kode:runcodeafter]
if $TIMEOUT_EXISTS; then
	sudo -u restricted_user timeout -s9 $((TIMELIMITINT*2)) $CMD <$IN >out 2>err
else
	sudo -u restricted_user $CMD <$IN >out 2>err        
fi
\end{lstlisting}

dan \textit{uncomment} baris yang ditunjukan pada Kode \ref{kode:runcodeuncomment}.

\begin{lstlisting}[language=C,caption=Kode \textit{runcode.sh} awal, label=kode:runcodeuncomment]
sudo -u restricted_user pkill -9 -u restricted_user
\end{lstlisting}

\item Mematikan akses jaringan untuk \textit{restricted\char`_user}

Pengguna dapat mematikan akses jaringan untuk \textit{restricted\char`_user} di \textit{linux} menggunakan \textit{iptables}. Setelah dimatikan lakukan pengujian menggunakan \verb|ping| sebagai \textit{restricted\char`_user}.

\item Menolak izin menulis

Pastikan tidak ada direktori atau \textit{file} yang dapat ditulis oleh \textit{restricted\char`_user}.

\item Membatasi jumlah proses

Pengguna dapat membatasi jumlah proses dari \textit{restricted\char`_user} dengan menambahkan melalui \textit{file /etc/security/limits.conf} seperti yang ditunjukan pada Kode \ref{kode:limitprocess}.

\begin{lstlisting}[caption=Contoh kode untuk membatasi jumlah proses, label=kode:limitprocess]
restricted_user     soft    nproc   3
restricted_user     hard    nproc   5
\end{lstlisting}

\end{itemize}

\item \subsubsection{Menggunakan dua \textit{server}}
Pengguna dapat menggunakan dua \textit{server} untuk antar muka web, menangani permintaan web, dan mengguna \textit{server} lainnya untuk menjalankan kode yang sudah dikumpulkan. Penggunaan dua \textit{server} mengurangi risiko menjalankan kode yang sudah dikumpulkan. Pengguna harus mengganti \textit{source code SharIF Judge} untuk memakai hal ini.

\end{enumerate}

\subsection{\textit{Sandboxing}}
\label{subsec:sandboxing}
\textit{SharIF Judge} menjalankan banyak \textit{arbitrary codes} yang pengguna kumpulkan. \textit{SharIF Judge} harus menjalankan kode pada lingkungan terbatas sehingga memerlukan perkakas untuk melakukan \textit{sandbox} kode yang sudah dikumpulkan. Pengguna dapat meningkatkan keamanan dengan menghidupkan \textit{shield} bersama dengan \textit{Sandbox}.

\subsubsection{\textit{C/C++ Sandboxing}}
\textit{SharIF Judge} menggunakan \textit{EasySandbox} untuk melakukan \textit{sandboxing} kode \textit{C/C++}. \textit{EasySandbox} berguna untuk membatasi kode yang berjalan menggunakan \textit{seccomp}. \textit{Seccomp} merupakan mekanisme \textit{sandbox} pada \textit{Linux kernel}. Secara \textit{default} \textit{EasySandbox} dimatikan pada \textit{SharIF Judge}, namun pengguna dapat menghidupkannya melalui halaman \textit{Settings}. Selain itu, pengguna juga harus melakukan \textit{"build EasySandbox"} sebelum menyalakannya. Berikut merupakan cara melakukan \textit{build EasySandbox}:

\textit{File EasySandbox} terdapat pada direktori \texttt{tester/easysandbox}. Untuk membangun \textit{EasySandbox} jalankan:

\begin{lstlisting}[caption=Kode \textit{runcode.sh} awal, label=kode:easysandbox]
$ cd tester/easysandbox
$ chmod +x runalltests.sh
$ chmod +x runtest.sh
$ make runtests
\end{lstlisting}

Jika keluaran berupa \textit{message} \texttt{All test passed!} maka, \textit{EasySandbox} berhasil dibangun dan dapat dinyalakan pada perangkat lunak.

\subsubsection{\textit{Java Sandboxing}}
Secara \textit{default}, \textit{Java Sandbox} sudah dinyalakan menggunakan \textit{Java Security Manager}. Pengguna dapat menghidupkan atau mematikannya pada halaman \textit{Settings}.

\subsection{\textit{Shield}}
\label{subsec:shield}
\textit{Shield} merupakan mekanisme sangat simpel untuk melarang jalannya kode yang berpotensi berbahaya. \textit{Shield} bukan solusi \textit{sandboxing} karena \textit{shield} hanya menyediakan proteksi sebagian terhadap serangan kecil. Proteksi utama terhadap kode tidak terpercaya adalah dengan menghidupkan \textit{Sandbox} yang dapat dilihat pada subbab \ref{subsec:sandboxing}.

\subsubsection{\textit{Shield} untuk \textit{C/C++}}
Dengan menghidupkan \textit{shield} untuk \textit{c/c++}, \textit{SharIF Judge} hanya perlu menambahkan \verb|#define| pada awal kode yang dikumpulkan sebelum menjalankannya. Sebagai contoh, pengguna dapat melarang penggunaan \verb|goto| dengan menambahkan Kode \ref{kode:goto} pada awal kode yang sudah dikumpulkan.
\begin{lstlisting}[language=C,caption=Kode \textit{shield} untuk melarang penggunaan goto, label=kode:goto]
#define goto YouCannotUseGoto
\end{lstlisting}
Dengan Kode \ref{kode:goto}, semua kode yang menggunakan \verb|goto| akan mendapatkan \textit{compilation error}. Apabila pengguna menghidupkan \textit{shield}, semua kode yang mengandung \verb|#undef| akan mendapatkan \textit{compilation error}. \textit{Shield} untuk \textit{C} dan \textit{C++} dapat dikonfigurasikan menggunakan cara berikut:

\begin{itemize}
\item Menghidupkan \textit{shield} untuk \textit{C/C++}

Pengguna dapat menghidupkan atau mematikan \textit{shield} pada halaman \textit{settings}.

\item Menambahkan aturan untuk \textit{C/C++}\\
Daftar aturan \verb|#define| untuk bahasa \textit{C} terdapat pada halaman \texttt{tester/shield/defc.h} dan \texttt{tester/shield/defcpp.h} untuk bahasa \textit{C++}. Pengguna dapat menambahkan aturan baru pada \textit{file} tersebut pada halaman \textit{settings}. Kode ref{kode:define} menunjukan contoh sintaks pada untuk menambahkan aturan \textit{shield}.

\begin{lstlisting}[language=C,caption=Sintaks aturan \textit{\#define}, label=kode:define]
/*

@file defc.h
There should be a newline at end of this file.
Put the message displayed to user after // in each line

e.g. If you want to disallow goto, add this line:
#define goto errorNo13    //Goto is not allowd

*/

#define system errorNo1      //"system" is not allowed
#define freopen errorNo2     //File operation is not allowed
#define fopen errorNo3       //File operation is not allowed
#define fprintf errorNo4     //File operation is not allowed
#define fscanf errorNo5      //File operation is not allowed
#define feof errorNo6        //File operation is not allowed
#define fclose errorNo7      //File operation is not allowed
#define ifstream errorNo8    //File operation is not allowed
#define ofstream errorNo9    //File operation is not allowed
#define fork errorNo10       //Fork is not allowed
#define clone errorNo11      //Clone is not allowed
#define sleep errorNo12      //Sleep is not allowed
\end{lstlisting}

Pada akhir baris \textit{file} \verb|defc.h| dan \verb|defcpp.h| harus terdapat baris baru. Terdapat banyak aturan yang tidak dapat digunakan pada \textit{g++}, seperti aturan \verb|#define fopen errorNo3| untuk bahasa \textit{C++} karena akan mengasilkan \texttt{compile error}.

\end{itemize} 
\subsubsection{\textit{Shield} untuk \textit{Python}}

Penggunaan \textit{shield} untuk \textit{python} dapat dihidupkan melalui halaman \textit{settings}. Dengan menghidupkan \textit{shield} untuk \textit{python}, \textit{SharIF Judge} hanya menambahkan beberapa kode pada baris awal kode yang sudah dikumpulkan sebelum dijalankan. Penambahan kode digunakan untuk mencegah pemakaian fungsi berbahaya. Kode-kode tersebut terletak pada \textit{file tester/shield/shield\_py2.py} dan \texttt{tester/shield/shield\_py3.py}. Kode \ref{kode:pythonshield} menunjukan cara untuk keluar dari \textit{shield} untuk \textit{python} menggunakan fungsi yang telah di daftar hitamkan.

\begin{lstlisting}[language=Python,caption=Cara keluar dari \textit{shield} untuk \textit{python},label=kode:pythonshield]

# @file shield_py3.py

import sys
sys.modules['os']=None

BLACKLIST = [
  #'__import__', # deny importing modules
  'eval', # eval is evil
  'open',
  'file',
  'exec',
  'execfile',
  'compile',
  'reload',
  #'input'
  ]
for func in BLACKLIST:
  if func in __builtins__.__dict__:
    del __builtins__.__dict__[func]
    
\end{lstlisting}

\section{CodeIgniter 4\cite{codeigniter:23:ci4}}
\label{sec:ci4}

\textit{CodeIgniter 4} merupakan versi terbaru dari \textit{framework} \textit{CodeIgniter} yang berfungsi untuk membantu pembentukan situs web. \textit{CodeIgniter 4} dapat dipasang menggunakan \textit{composer} ataupun dipasang secara manual dengan mengunduhnya dari situs web resmi. Berikut merupakan sintaks untuk melakukan pemasangan menggunakan \textit{composer}.

\begin{center}
\verb|composer create-project codeigniter4/appstarter project-root|
\end{center}

Sintaks diatas akan mengunduh dan melakukan instalasi projek \textit{CodeIgniter 4} dengan nama \texttt{project-root}. Sintaks \texttt{codeigniter4/appstarter} berfungsi untuk mengunduh aplikasi \textit{skeleton} dari projek \textit{CodeIgniter 4} yang berisikan kebutuhan data untuk melakukan pembangunan sebuah aplikasi. Setelah dilakukan pemasangan, \textit{CodeIgniter 4} memiliki lima buah direktori dengan struktur aplikasi sebagai berikut:

\begin{itemize}
\item \texttt{app/}

Direktori \textit{app} berisikan semua kode yang dibutuhkan untuk menjalankan aplikasi web yang dibentuk. Direktori ini terdiri dari beberapa direktori yang terletak didalamnya sebagai berikut:
\begin{itemize}
\item \verb|Config/| berfungsi dalam menyimpan \textit{file} konfigurasi aplikasi web pengguna.
\item \verb|Controllers/| berfungsi sebagai penentu alur program yang dibentuk.
\item \verb|Database/| berfungsi sebagai penyimpan \textit{file migrations} dan \textit{seeds}.
\item \verb|Filters/| befungsi dalam menyimpan \textit{file} kelas \textit{filter}.
\item \verb|Helpers/| berfungsi dalam menyimpan koleksi fungsi mandiri.
\item \verb|Language/| berfungsi sebagai tempat penyimpanan \textit{string} dalam berbagai bahasa.
\item \verb|Libraries/| berfungsi dalam penyimpan kelas yang tidak termasuk kategori lain.
\item \verb|Models/| berfungsi untuk merepresentasikan data dari \textit{database}.
\item \verb|ThirdParty/| \textit{library ThirdParty} yang dapat digunakan pada aplikasi.
\item \verb|Views/| berisikan \textit{file} HTML yang ditampilkan kepada pengguna.
\end{itemize}

\item \texttt{public/}

Direktori \textit{public} merupakan akar dari situs web dan berisikan data-data yang dapat diakses oleh pengguna melalui \textit{browser}. Direktori ini berisikan \textit{file} \verb|.htaccess|, \verb|index.php|, dan seluruh \textit{assets} dari aplikasi \textit{website} yang dibentuk seperti gambar, CSS, ataupun \textit{JavaScript}.

\item \texttt{writable/}

Direktori \textit{writable} berisikan data-data yang mungkin perlu ditulis oleh perangkat lunak seperti \textit{file cache}, \textit{logs}, dan \textit{uploads}. Pengguna dapat menambahkan direktori baru sesuai dengan kebutuhan sehingga menambahkan keamanan pada direktori utama.

\item \texttt{tests/}

Direktori ini menyimpan \textit{file test} dan tidak perlu dipindahkan ke \textit{server} produksi. Direktori ini terdapat direktori \verb|_support| berisikan berbagai jenis kelas \textit{mock} dan keperluan yang dapat dipakai pengguna dalam menulis \textit{tests}.

\item \texttt{vendor/} atau \texttt{system/}

Direktori ini berisikan \textit{file} yang diperlukan dalam menjalani \textit{framework} dan tidak boleh dimodifikasi oleh pengguna. Pengguna dapat melakukan \textit{extend} atau membangun kelas baru untuk menambahkan fungsi yang diperlukan.

\end{itemize}

\subsection{\textit{Models-Views-Controllers}}
\textit{CodeIgniter 4} menggunakan struktur \textit{MVC} untuk mengatur \textit{file} agar lebih simpel dalam menemukan \textit{file} yang diperlukan. MVC menyimpan data, presentasi, dan alur program dalam \textit{file} yang berbeda.

\subsubsection{\textit{Models}}

\textit{Models} berfungsi dalam menyimpan dan mengambil data dari tabel spesifik pada \textit{database}. Data tersebut dapat berupa pengguna, pemberitahuan blog, transaksi, dll. \textit{Models} pada umumnya disimpan pada direktori \texttt{app/Models} dan memiliki \textit{namespace} sesuai dengan lokasi dari \textit{file} tersebut. Kode \ref{kode:modelsexample} menunjukan contoh dari \textit{models} bernama \texttt{UserModel}.

\begin{lstlisting}[language=PHP, caption=Contoh \textit{Models},label=kode:modelsexample]

<?php

namespace App\Models;

use CodeIgniter\Model;

class UserModel extends Model
{
    protected $table      = 'users';
    protected $primaryKey = 'id';

    protected $useAutoIncrement = true;

    protected $returnType     = 'array';
    protected $useSoftDeletes = true;

    protected $allowedFields = ['name', 'email'];

    // Dates
    protected $useTimestamps = false;
    protected $dateFormat    = 'datetime';
    protected $createdField  = 'created_at';
    protected $updatedField  = 'updated_at';
    protected $deletedField  = 'deleted_at';

    // Validation
    protected $validationRules      = [];
    protected $validationMessages   = [];
    protected $skipValidation       = false;
    protected $cleanValidationRules = true;

}
    
\end{lstlisting}

Kode \ref{kode:modelsexample} merupakan contoh \textit{file} \textit{Models} yang dapat digunakan pada \textit{controllers}. \textit{Models} tersebut terkoneksikan dengan tabel \texttt{users} dengan \textit{primarykey} \texttt{id}. \textit{Model} pada \textit{CodeIgniter 4} juga dapat digunakan untuk mencari, menyimpan, dan menghapus data untuk setiap tabel spesifik. Kode \ref{kode:modelfindall} menunjukan contoh penggunaan \textit{model} dengan menggabungkan \textit{query} dengan metode pencarian \textit{model} untuk mencari seluruh data \verb|active|.
\begin{lstlisting}[language=PHP, caption=Contoh penggunaan \textit{model} untuk mencari data spesifik,label=kode:modelfindall]
<?php

$users = $userModel->where('active', 1)->findAll();
\end{lstlisting}

\subsubsection{\textit{Views}}

\textit{Views} merupakan halaman berisikan HTML dan sedikit PHP yang ditampilkan kepada pengguna. \textit{Views} juga dapat berupa pecahan halaman seperti \textit{header, footer}, ataupun \textit{sidebar}. \textit{Views} biasanya terdapat pada \verb|app/Views| dan mendapatkan data berupa variabel dari \textit{controller} untuk ditampilkan.

\begin{lstlisting}[language=PHP, caption=Contoh \textit{Views},label=kode:viewsexample]
<html>
    <head>
        <title>My Blog</title>
    </head>
    <body>
        <h1>Welcome to my Blog!</h1>
    </body>
</html>
\end{lstlisting}

Kode \ref{kode:viewsexample} menunjukan contoh \textit{view} pada direktori \verb|app/Views| yang akan menampilkan tulisan \texttt{Welcome to my Blog}. \textit{View} ini dapat ditampilkan melalui controller seperti yang ditunjukan pada Kode \ref{kode:viewsexamplecontroller}.
\begin{lstlisting}[language=PHP, caption=Contoh menampilkan \textit{Views} pada \textit{controller},label=kode:viewsexamplecontroller]
<?php

namespace App\Controllers;

use CodeIgniter\Controller;

class Blog extends Controller
{
    public function index()
    {
        return view('blog_view');
    }
}
\end{lstlisting}

Kode \ref{kode:viewsexamplecontroller} merupakan contoh memanggil \textit{views} pada \textit{file controllers}. Kode ini mengembalikan halaman \texttt{blog\_view} pada \textit{controller} dengan nama \texttt{Blog}.

\subsubsection{\textit{Controllers}}

\textit{Contollers} merupakan kelas untuk mengambil atau memberikan data dari \textit{models} menuju \textit{views} untuk ditampilkan kepada pengguna. Setiap pembentukan \textit{controllers} dibentuk harus memperpanjang kelas \texttt{BaseController}. Kode \ref{kode:controllersexample} menunjukan contoh \textit{controllers} yang dibentuk dengan nama \texttt{Helloworld}.

\begin{lstlisting}[language=PHP, caption=Contoh \textit{Controllers} pada \textit{CodeIgniter 4},label=kode:controllersexample]
<?php

namespace App\Controllers;

class Helloworld extends BaseController
{
    public function index()
    {
        return 'Hello World!';
    }

    public function comment()
    {
        return 'I am not flat!';
    }
}
\end{lstlisting}
Kode \ref{kode:controllersexample} akan mengembalikan kata \textit{Hello World} pada url:
\begin{center}
\verb|example.com/index.php/helloworld/|
\end{center}

Selain itu, \textit{CodeIgniter 4} menyediakan fungsi bernama \textit{Controller Filters} dan kelas \textit{IncomingRequest}. \textit{Controller Filter} memiliki fungsi untuk membiarkan pengguna membangun sebuah kondisi sebelum ataupun sesudah \textit{controller} dijalankan. Kode \ref{kode:filterexample} menunjukan contoh pembentukan sebuah \textit{filters} dengan nama \texttt{MyFilter}.

\begin{lstlisting}[language=PHP, caption=Contoh \textit{Controllers Filters} pada \textit{CodeIgniter 4},label=kode:filterexample]
<?php

namespace App\Filters;

use CodeIgniter\Filters\FilterInterface;
use CodeIgniter\HTTP\RequestInterface;
use CodeIgniter\HTTP\ResponseInterface;

class MyFilter implements FilterInterface
{
    public function before(RequestInterface $request, $arguments = null)
    {
        $auth = service('auth');

        if (! $auth->isLoggedIn()) {
            return redirect()->to(site_url('login'));
        }
    }
}
\end{lstlisting}

Kode \ref{kode:filterexample} berfungsi contoh kode yang akan melakukan pengecekan apakah pengguna sudah melakukan \textit{login} sebelum \textit{controller} dijalankan. Kode ini akan disimpan pada direktori \texttt{app/Filters}. Selanjutnya pengguna perlu menambahkan konfigurasi \textit{filter} pada \textit{routes}. Sintaks dibawah akan melakukan pengecekan pada \textit{controller} \texttt{Dashboard::index} sebelum dan setelah \textit{controller} tersebut dijalankan menggunakan \textit{filter} dengan nama \texttt{admin-auth}.
\begin{center}
	\verb|$routes->get('/', 'Dashboard::index',['filter' => 'admin-auth:dual,noreturn']);|
\end{center}

Selanjutnya kelas \textit{IncomingRequest} menyediakan representasi \textit{object-oriented} sebuah \textit{HTTP request} dari \textit{client} seperti \textit{browser}. Kode \ref{kode:incomingreqbab2} menunjukan contoh \textit{IncomingRequest} untuk mengakses data yang telah dikirimkan oleh \textit{client}.
\begin{lstlisting}[language=PHP, caption=Contoh mengakses data menggunakan \textit{IncomingRequest},label=kode:incomingreqbab2]
<?php

namespace App\Controllers;

use CodeIgniter\Controller;

class UserController extends Controller
{
    public function index()
    {
        if ($this->request->isAJAX()) {
            $something = $this->request->getVar('foo');
        }
    }
}
\end{lstlisting}
Sintaks \texttt{isAJAX} pada Kode \ref{kode:incomingreqbab2} berfungsi untuk melakukan pengecekan apakan data yang dikirimkan melalui \textit{AJAX}. Sedangkan sintaks \texttt{getVar('foo')} berfungsi untuk mengambil data yang telah dikirimkan dengan nama \textit{foo}. Selain itu apabila pengguna ingin mengakses kelas ini diluar \textit{controller}, pengguna dapat menginisiasikan kelas ini menggunakan sintaks berikut:
\begin{center}
	\verb|$request = \Config\Services::request();|
\end{center}
Sintaks diatas akan menginisiasikan kelas \textit{request} dan menyimpannya menuju sebuah variabel yang dapat diakses oleh seluruh fungsi pada kelas tersebut.
\subsection{\textit{Autoloading Files}}
\textit{CodeIgniter 4} menyediakan fitur \textit{autoloader} yang dapat digunakan dengan sedikit konfigurasi. Fitur ini dapat menemukan \textit{namespaced classes} individual yang menggunakan struktur direktori \textit{autoloading} pada PSR-4. Fitur ini juga dapat digunakan bersamaan dengan \textit{autoloader} lain seperti \textit{composer}. Konfigurasi fitur ini dapat dilakukan pada direktori \texttt{app/Config/Autoload.php} yang berisikan dua buah \textit{array} utama yakni \texttt{classmap} dan \texttt{psr4}. Kode \ref{kode:autoloadnamespace} menunjukan contoh konfigurasi menggunakan \textit{namespace PSR-4} untuk melakukan \textit{mapping} menuju direktori.
\begin{lstlisting}[language=PHP, caption=Contoh konfigurasi menggunakan \textit{namespace PSR-4}. ,label=kode:autoloadnamespace]
<?php

namespace Config;

use CodeIgniter\Config\AutoloadConfig;

class Autoload extends AutoloadConfig
{
    // ...
    public $psr4 = [
        APP_NAMESPACE => APPPATH, // For custom app namespace
        'Config'      => APPPATH . 'Config',
    ];

    // ...
}
\end{lstlisting}
\textit{Key} dari setiap baris merupakan \textit{namespace} itu sendiri sedangkan \textit{value} dari \textit{array} merupakan \textit{path} dari direktori. Pengguna dapat melakukan pengecekan terhadap konfigurasi \textit{namespace} menggunakan sintaks sebagai berikut:
\begin{center}
	\verb|php spark namespaces|
\end{center}
Sintaks diatas dapat dijalankan melalui \textit{command line} pada aplikasi. Konfigurasi selanjutnya menggunakan \textit{classmap} untuk menghubungkan terhadap \textit{third-party libraries} yang tidak memiliki \textit{namespace}. Kode \ref{kode:autoloadclassmap} menunjukan contoh konfigurasi menggunakan \textit{classmap}. \textit{Key} pada \textit{array} \texttt{classmap} merupakan nama kelas yang ingin ditemukan sedangkan \textit{value} merupakan \textit{path} dari kelas itu sendiri.

\begin{lstlisting}[language=PHP, caption=Contoh konfigurasi menggunakan \textit{classmap}. ,label=kode:autoloadclassmap]
<?php

namespace Config;

use CodeIgniter\Config\AutoloadConfig;

class Autoload extends AutoloadConfig
{
    // ...
    public $classmap = [
        'Markdown' => APPPATH . 'ThirdParty/markdown.php',
    ];

    // ...
}
\end{lstlisting}

\subsection{\textit{Configuration}}
Konfigurasi pada \textit{CodeIgniter 4} terletak pada direktori \texttt{app/Config}. Konfigurasi pada \textit{CodeIgniter 4} tidak ditempatkan pada satu buah \textit{file} melainkan setiap kelas yang membutuhkan konfigurasi memiliki \textit{file} yang berbeda. Pengguna dapat mengakses \textit{file} \textit{configuration} dengan beberapa cara sebagai berikut:
\begin{lstlisting}[language=PHP, caption=Contoh mengakses file \textit{configuration}. ,label=kode:configbab2ci4]
<?php
// Creating new configuration object by hand
$config = new \Config\Pager();

// Get shared instance with config function
$config = config('Pager');

// Access config class with namespace
$config = config('Config\\Pager');
$config = config(\Config\Pager::class);

// Creating a new object with config function
$config = config('Pager', false);
\end{lstlisting}
Kode \ref{kode:configbab2ci4} menunjukan beberapa contoh untuk mengakses file \textit{configuration}. Pengguna dapat mengakses \textit{configuration} secara manual menggunakan sintaks \texttt{new}, pengguna dapat mengakses \textit{configuration} menggunakan fungsi \textit{config} dengan sintaks \texttt{config('namafile')}, menggunakan \textit{namespace}, dan membentuk objek baru menggunakan fungsi \textit{config}. Selanjutnya pengguna dapat mengakses properti yang terdapat pada file \textit{config} tersebut menggunakan sintaks berikut:
\begin{center}
	\verb|$pageSize = $config->perPage;|
\end{center}
Sintaks diatas akan mengambil properti \texttt{perPage} yang terdapat pada variabel \textit{config} yang sudah diinisiasi. Selain menggunakan \textit{config} yang terdapat pada \textit{CodeIgniter 4}, pengguna dapat membentuk file \textit{config} secara manual. Kode \ref{kode:configbab2ci4new} menunjukan contoh isi \textit{file} \textit{config} yang dibentuk manual.
\begin{lstlisting}[language=PHP, caption=Contoh pembentukan file \textit{configuration}. ,label=kode:configbab2ci4new]
<?php

namespace Config;

use CodeIgniter\Config\BaseConfig;

class CustomClass extends BaseConfig
{
    public $siteName  = 'My Great Site';
    public $siteEmail = 'webmaster@example.com';
    // ...
}
\end{lstlisting}
\textit{File config} ini akan disimpan pada direktori \texttt{app/Config} dengan nama kelas yang akan \texttt{extends} dengan \texttt{BaseConfig}. Selain menggunakan file \textit{config}, pengguna juga dapat melakukan konfigurasi menggunakan variabel \textit{environment}. Penggunaan file \textit{environment} merupakan cara melakukan konfigurasi terbaik untuk saat ini dikarenakan kemudahan untuk mengubah konfigurasi pada saat melakukan \textit{deploy} aplikasi. Konfigurasi dapat diubah tanpa harus mengubah kode. \textit{CodeIgniter 4} menyediakan sebuah file bernama \textit{dotenv} atau yang akan disebut \texttt{.env} selanjutnya. \texttt{.env} merupakan sebuah file yang terletak pada akar aplikasi. File ini dapat berisikan seluruh konfigurasi yang diperlukan oleh aplikasi. Berikut merupakan contoh variabel yang disimpan pada file \texttt{.env}.
\begin{lstlisting}[caption=Contoh variabel yang disimpan pada file \texttt{.env}. ,label=kode:configbab2env]
S3_BUCKET = dotenv
SECRET_KEY = super_secret_key
CI_ENVIRONMENT = development
\end{lstlisting}
Kode \ref{kode:configbab2env} menunjukan contoh variabel konfigurasi yang disimpan pada \textit{file} \texttt{.env}. Variabel dapat berisikan konfigurasi yang bersifat pribadi seperti \textit{password} dan \textit{API keys}. Namun, konfigurasi hanya bersifat sebagai pengganti data sehingga pengguna tidak dapat membentuk data baru melainkan hanya mengubah data yang sudah ada sebelumnya.

\subsection{\textit{CodeIgniter URLs}}
\textit{CodeIgniter 4} menggunakan pendekatan \textit{segment-based} dibandingkan menggunakan \textit{query-string} untuk menghasilkan URL sehingga ramah manusia dan mesin pencari. Berikut merupakan contoh \textit{URL} yang dihasilkan oleh \textit{CodeIgniter 4}:

\begin{center}
\texttt{https://www.example.com/ci-blog/blog/news/2022/10?page=2}
\end{center}

\textit{CodeIgniter 4} menghasilkan \textit{URL} seperti diatas dengan membaginya menjadi beberapa bagian sebagai berikut:

\begin{itemize}
\item URL dasar merupakan URL dasar dari aplikasi web yang ditunjukan sebagai \texttt{baseURL}. URL dasar ini berbentuk \texttt{https://www.example.com/ci-blog/}.
\item \textit{URI Path} merupakan alamat yang dituju yaitu \texttt{/ci-blog/blog/news/2022/10} 
\item \textit{Route} juga merupakan alamat yang dituju tanpa URL dasar yaitu \texttt{/blog/news/2022/10} 
\item \textit{Query} merupakan hasil dari \textit{query} yang ingin ditampilkan yaitu \texttt{page=2}
\end{itemize}

Secara \textit{default}, \textit{CodeIgniter 4} membangun \textit{URL} dengan \verb|index.php| namun, pengguna dapat menghapus \textit{file} \verb|index.php| pada URL yang dibentuk. Pengguna dapat menghapus \verb|index.php| sesuai dengan \textit{server} yang digunakan. Berikut merupakan contoh dua buah \textit{server} yang umum dipakai:

\subsubsection{\textit{Apache Web Server}}
Pengguna dapat URL melalui \textit{file} \verb|.htaccess| dengan menyalakan ekstensi \verb|mod_rewrite|. Kode \ref{kode:htaccessapache} menunjukan contoh \textit{file} \verb|.htaccess| untuk menghapus \verb|index.php| pada \textit{URL} yang dibentuk.

\begin{lstlisting}[caption=Contoh \textit{file} \texttt{.htaccess} pada \textit{Apache Web Server}. ,label=kode:htaccessapache]
RewriteEngine On
RewriteCond %{REQUEST_FILENAME} !-f
RewriteCond %{REQUEST_FILENAME} !-d
RewriteRule ^(.*)$ index.php/$1 [L]
\end{lstlisting}

Kode \ref{kode:htaccessapache} memperlakukan semua \textit{HTTP Request} selain dari direktori dan \textit{file} yang ada sebagai permintaan \textit{file} \texttt{index.php}.

\subsubsection{\textit{NGINX}}
Pengguna dapat mengubah URL menggunakan \verb|try_files| yang akan mencari URI dan mengirimkan permintaan pada URL yang ingin dihilangkan. Kode \ref{kode:tryfilesnginx} menunjukan contoh penggunaan \verb|try_files| untuk menghapus \verb|index.php| pada URL.

\begin{lstlisting}[caption=Contoh penggunaan \texttt{try-files}. ,label=kode:tryfilesnginx]
location / {
    try_files $uri $uri/ /index.php$is_args$args;
}
\end{lstlisting}
Kode \ref{kode:tryfilesnginx} akan mencari file atau direktori sesuai dengan URI yang diterima. Selanjutnya \textit{request} akan dikirimkan menuju file \texttt{index.php} bersamaan dengan argumen yang dikirimkan.

\subsection{\textit{Error Handling}}
\textit{CodeIgniter 4} melaporkan \textit{error} melalui \textit{exceptions}. Laporan ini akan dikirimkan sesuai dengan konfigurasi \textit{environment} pengguna kecuali \textit{environment production}. Kode \ref{kode:errorhandlingbab2} menunjukan cara mengirimkan pesan \textit{error} kepada pengguna.
\begin{lstlisting}[language=PHP, caption=Contoh penggunaan \texttt{error handling}. ,label=kode:errorhandlingbab2]
throw new \Exception('Some message goes here');
\end{lstlisting}
Kode \ref{kode:errorhandlingbab2} akan mengirimkan pesan berupa \texttt{Some message goes here} kepada pengguna pada \textit{environment development} dan \textit{testing}. Selain itu pengguna dapat mengambil \textit{error} apabila terdapat sebuah method yang memungkinkan untuk dikirimkan sebuah \textit{exception}. Kode \ref{kode:errorhandlingtrybab2} menunjukan cara untuk mengambil \textit{exception} pada sebuah method.
\begin{lstlisting}[language=PHP, caption=Contoh penggunaan \texttt{error handling} untuk mengambil \textit{exception}. ,label=kode:errorhandlingtrybab2]
try {
    $user = $userModel->find($id);
} catch (\Exception $e) {
    exit($e->getMessage());
}
\end{lstlisting}
Kode \ref{kode:errorhandlingtrybab2} akan mengirimkan pesan sesuai dengan yang telah ditentukan oleh pengguna apabila method yang dijalankan terdapat sebuah \textit{error}. Selain menentukan pesan \textit{error}, pengguna juga dapat memberikan \textit{error} berupa 404 seperti yang ditunjukan pada Kode \ref{kode:errorhandling404bab2}.
\begin{lstlisting}[language=PHP, caption=Contoh penggunaan \texttt{error handling} untuk memberikan \textit{error} berupa 404. ,label=kode:errorhandling404bab2]
if (! $page = $pageModel->find($id)) {
    throw \CodeIgniter\Exceptions\PageNotFoundException::forPageNotFound();
}
\end{lstlisting}
Kode \ref{kode:errorhandling404bab2} merupakan contoh sintaks untuk memberikan sebuah \textit{error} berupa 404 atau halaman tidak ditemukan. Kode \ref{kode:errorhandling404bab2} akan melakukan pengecekan terhadap \textit{model} dan akan memberikan \textit{error} apabila kondisi tidak ditemukan. Pengguna juga dapat memberikan pesan pada \textit{exception} tersebut yang dapat ditampilkan pada \textit{environment development} dan \textit{testing}.

\subsection{\textit{URI Routing}}
\textit{CodeIgniter 4} menyediakan fitur URI \textit{routing} yang berfungsi untuk menentukan \textit{route} dari setiap URL dan \textit{controller} yang diinginkan. \textit{CodeIgniter 4} menyediakan dua buah cara \textit{routing} yang dapat digunakan sebagai berikut:
\subsubsection{\textit{Defined Route Routing}}
Pengguna dapat mendefinisikan \textit{route} secara manual untuk \textit{URL} yang lebih fleksibel. Kode \ref{kode:ci4route1} menunjukan contoh \textit{route} yang didefinisikan secara manual untuk membentuk URL menuju kelas \texttt{Catalog} dengan metode \texttt{productLookup}.
\begin{lstlisting}[language=PHP,caption=Contoh \textit{route} yang didefinisikan secara manual,label=kode:ci4route1]
<?php

$routes->get('product/(:num)', 'Catalog::productLookup');
\end{lstlisting}
Pengguna juga dapat memakai beberapa HTTP \textit{verb} seperti \textit{GET, POST}, dan \textit{PUT}. Selain menulis \textit{route} secara individu, pengguna dapat melakukan \textit{grouping} pada route seperti yang ditunjukan Kode \ref{kode:ci4routegroup}.
\begin{lstlisting}[language=PHP,caption=Contoh \textit{route} yang menggunakan \textit{grouping} manual,label=kode:ci4routegroup]
<?php

$routes->group('admin', static function ($routes) {
    $routes->get('users', 'Admin\Users::index');
    $routes->get('blog', 'Admin\Blog::index');
});
\end{lstlisting}

Kode \ref{kode:ci4routegroup} menunjukan contoh penggunaan \textit{grouping} untuk URI \texttt{admin/users} dan \texttt{admin/blog}. Fitur ini memudahkan pengguna untuk menentukan \textit{route} pada setiap fungsi yang ingin dibentuk pada \textit{controller} \texttt{admin}.

\subsubsection{\textit{Auto Routing}}
\label{subsubsec:autorouting}
Pengguna dapat mendefinisikan \textit{route} secara otomatis melalui fitur \textit{Auto Routing} apabila tidak terdapat \textit{route} yang didefinisikan. \textit{Auto Routing} membentuk URI berdasarkan nama dan fungsi pada \textit{controller} yang dituju. Pengguna dapat menyalakan fitur ini pada \texttt{app/Config/Routes.php} dengan cara sebagai berikut:
\begin{center}
	\verb|public bool $autoRoute = true;|
\end{center}
Pengguna juga perlu mengubah \verb|$autoRoutesImproved| menjadi \verb|true| pada file \verb|app/Config/Feature.php|. Selain menggunakan \textit{auto routing} baru, pengguna dapat menggunakan \textit{Auto Routing (Legacy)} seperti yang terdapat pada \textit{CodeIgniter 3} dengan cara seperti berikut:
\begin{center}
\verb|$routes->setAutoRoute(true);|
\end{center}
Penggunaan \textit{Auto Routing (Legacy)} tidak disarankan karena fitur ini melewatkan pengecekan \textit{controller filters} dan proteksi CSRF.

\subsection{\textit{Database}}
\textit{CodeIgniter 4} menyediakan kelas \textit{database} yang dapat menyimpan, memasukan, memperbarui, dan menghapus data pada \textit{database} sesuai dengan konfigurasi. Pengguna dapat melakukan konfigurasi untuk \textit{database} yang ingin dikoneksikan melalui direktori \verb|app/Config/Database.php| atau \textit{file} \verb|.env|. Kode \ref{kode:databaseconfci4} menunjukan contoh konfigurasi untuk database bernama \verb|database_name| dengan \textit{username} root.

\begin{lstlisting}[language=PHP, caption=Contoh konfigurasi \textit{database} pada \textit{CodeIgniter 4}. ,label=kode:databaseconfci4]
<?php

namespace Config;

use CodeIgniter\Database\Config;

class Database extends Config
{
    public $default = [
        'DSN'      => '',
        'hostname' => 'localhost',
        'username' => 'root',
        'password' => '',
        'database' => 'database_name',
        'DBDriver' => 'MySQLi',
        'DBPrefix' => '',
        'pConnect' => true,
        'DBDebug'  => true,
        'charset'  => 'utf8',
        'DBCollat' => 'utf8_general_ci',
        'swapPre'  => '',
        'encrypt'  => false,
        'compress' => false,
        'strictOn' => false,
        'failover' => [],
        'port'     => 3306,
    ];

    // ...
}
\end{lstlisting}

Selain itu, konfigurasi juga dapat dilakukan pada file \texttt{.env} untuk mempermudah dalam pengubahan data pada saat melakukan \textit{deploy}. Kode \ref{kode:envdatabaseci4} menunjukan contoh konfigurasi pada file \texttt{.env}:
\begin{lstlisting}[caption=Contoh konfigurasi \textit{database} pada file \texttt{.env}. ,label=kode:envdatabaseci4]
database.default.username = 'root';
database.default.password = '';
database.default.database = 'ci4';
\end{lstlisting}

Kode \ref{kode:envdatabaseci4} akan menyimpan konfigurasi pada grup \textit{default} dengan \textit{username} berupa \texttt{root}, tanpa menggunakan \textit{password}, dan juga dengan nama \textit{database} ci4. Selain untuk melakukan koneksi \textit{database}, kelas ini dapat digunakan untuk menambahkan, menghapus, dan memperbaharui data pada \textit{database}. Kode \ref{kode:queryexampleci4} menunjukan contoh penggunaan \textit{query} untuk mengambil data \textit{title,content}, dan \textit{date} pada tabel \textit{users}.

\begin{lstlisting}[language=PHP, caption=Contoh penggunaan \textit{query} menggunakan konfigurasi pada \textit{CodeIgniter 4}. ,label=kode:queryexampleci4]
<?php

$builder = $db->table('users');
$builder->select('title, content, date');
$query = $builder->get();
\end{lstlisting}
\textit{CodeIgniter 4} juga menyediakan fitur untuk membangun \textit{database} melalui fitur bernama \textit{Database Forge}. Pengguna dapat membangun, mengubah, menghapus tabel dan juga menambahkan \textit{field} pada tabel tersebut. Kode \ref{kode:ci4databaseforge} menunjukan contoh pembentukan \textit{database}.
\begin{lstlisting}[language=PHP, caption=Contoh pembentukan tabel melalui \textit{database forge}. ,label=kode:ci4databaseforge]
<?php

$fields = [
    'id' => [
        'type'           => 'INT',
        'constraint'     => 5,
        'unsigned'       => true,
        'auto_increment' => true,
    ],
    'title' => [
        'type'       => 'VARCHAR',
        'constraint' => '100',
        'unique'     => true,
    ],
    'author' => [
        'type'       => 'VARCHAR',
        'constraint' => 100,
        'default'    => 'King of Town',
    ],
    'description' => [
        'type' => 'TEXT',
        'null' => true,
    ],
    'status' => [
        'type'       => 'ENUM',
        'constraint' => ['publish', 'pending', 'draft'],
        'default'    => 'pending',
    ],
];
$forge->addField($fields);
$forge->createTable('table_name');
\end{lstlisting}
Kode \ref{kode:ci4databaseforge} akan membentuk \textit{database} dengan tabel bernama \textit{table\_name} yang berisikan beberapa \textit{field} seperti \texttt{id} dan \textit{title}. 

\subsection{\textit{Library}}
\textit{CodeIgniter 4} menyediakan berbagai \textit{library} untuk membantu pengguna dalam pembentukan aplikasi \textit{website}. Berikut merupakan beberapa \textit{library} yang disediakan oleh \textit{CodeIgniter 4}:
\subsubsection{Kelas \textit{Email}}
\textit{CodeIgniter} menyediakan kelas \textit{email} dengan fitur sebagai berikut:
\begin{itemize}
\item Beberapa Protokol: \textit{Mail, Sendmail},dan SMTP
\item Enkripsi TLS dan SSL untuk SMTP
\item Beberapa Penerima
\item CC dan BCC
\item HTML atau \textit{email} teks biasa
\item Lampiran
\item Pembungkus kata
\item Prioritas
\item Mode BCC \textit{Batch}, memisahkan daftar \textit{email} besar menjadi beberapa BCC kecil.
\item Alat \textit{Debugging email}
\end{itemize}

Pengguna dapat melakukan konfigurasi pada \textit{file} \verb|app/Config/Email.php| untuk melakukan pengiriman \textit{email}. Kode \ref{kode:ci4emailclassconfig} menunjukan contoh konfigurasi preferensi pengiriman \textit{email} secara manual.
 \begin{lstlisting}[language=PHP, caption=Contoh kode untuk melakukan konfigurasi \textit{email}. ,label=kode:ci4emailclassconfig]
<?php

$config['protocol'] = 'sendmail';
$config['mailPath'] = '/usr/sbin/sendmail';
$config['charset']  = 'iso-8859-1';
$config['wordWrap'] = true;

$email->initialize($config);
\end{lstlisting}

Kode \ref{kode:ci4emailclassconfig} menentukan konfigurasi protokol agar menggunakan \texttt{sendmail}, \textit{server path} pada \textit{Sendmail}, pengggunaan karakter \textit{email}, dan menyalakan fitur \textit{wordwrap}. Selain itu, pengguna dapat melakukan pengiriman \textit{email} sesuai dengan kebutuhan. Kode \ref{kode:ci4emailclass} menunjukan contoh penggunaan kelas \textit{email} untuk mengirim \textit{email}.
\begin{lstlisting}[language=PHP, caption=Contoh kode untuk melakukan pengiriman \textit{email}. ,label=kode:ci4emailclass]
<?php

$email = \Config\Services::email();

$email->setFrom('your@example.com', 'Your Name');
$email->setTo('someone@example.com');
$email->setCC('another@another-example.com');
$email->setBCC('them@their-example.com');

$email->setSubject('Email Test');
$email->setMessage('Testing the email class.');

$email->send();
\end{lstlisting}
Kode \ref{kode:ci4emailclass} mengirimkan \textit{email} dari \texttt{your@example.com} kepada \texttt{someone@example.com} dengan subjek \texttt{Email Test} dan pesan \texttt{Testing the email class}. \textit{Email} juga akan dikirimkan melalui CC kepada \texttt{'another@another-example.com'} dan mengirimkan BCC menuju \texttt{'them@their-example.com'}.
\subsubsection{\textit{Session}}
\textit{Library session} memperbolehkan aplikasi untuk melihat jejak dari aktivitas dari pengguna aplikasi tersebut. Data \textit{session} dapat disimpan pada beberapa tempat yakni \textit{file}, \textit{database, memcached, redis}, dan \textit{array}. \textit{Session} dapat dilakukan inisiasi dengan sintaks sebagai berikut:
\begin{center}
	\verb|$session = \Config\Services::session($config);|
\end{center}
Sintaks diatas akan melakukan inisiasi \textit{library session} dan pengguna dapat memanggil \textit{library} melalui variabel \texttt{session}. \textit{Session} bekerja saat halaman aplikasi dari pengguna termuat. Aplikasi akan melakukan pengecekan terhadap \textit{session cookie} dan melakukan pembaharuan sesuai dengan \textit{session cookie} yang didapat. Pengguna juga dapat mengambil data yang tesimpan pada \textit{session} menggunakan sintaks berikut:
\begin{center}
	\verb|$item = $session->get('item');|
\end{center}
Sintaks diatas akan mengambil data \textit{session} dengan nama \textit{item} dan menyimpannya menuju sebuah variabel. Selain itu, \textit{session} juga dapat dihapus menggunakan sintaks sebagai berikut:
\begin{center}
	\verb|$session->destroy();|
\end{center}
Sintaks diatas akan membersihkan \textit{session} dari pengguna seperti saat melakukan \textit{logout}. Secara \textit{default session} akan menyimpan data menuju file namun, pengguna dapat melakukan konfigurasi sesuai dengan keperluan. Kode \ref{kode:ci4sessionclass} menunjukan konfigurasi untuk menyimpan data \textit{session} menuju \textit{database}.
\begin{lstlisting}[language=PHP, caption=Contoh kode untuk konfigurasi penyimpanan \textit{session}. ,label=kode:ci4sessionclass]
class Session extends BaseConfig
{
    // ...
    public string $driver = 'CodeIgniter\Session\Handlers\DatabaseHandler';

    // ...
    public string $savePath = 'ci_sessions';

    // ...
}
\end{lstlisting}
Kode \ref{kode:ci4sessionclass} akan mengubah penyimpanan menggunakan \texttt{databasehandler} dengan tabel penyimpanan bernama \texttt{ci\_sessions}. Konfigurasi dapat dilakukan pada direktori \texttt{app/Config} dengan nama \textit{file} \texttt{Session.php}.

\subsubsection{\textit{Working with Uploaded Files}}
\textit{CodeIgniter 4} menyediakan \textit{library} untuk melakukan pengunggahan \textit{file}. Pengunggahan \textit{file} terdapat empat buah proses sebagai berikut:
\begin{enumerate}
\item Dibentuk sebuah form untuk pengguna memilih dan mengunggah \textit{file}.
\item Setelah \textit{file} diunggah, \textit{file} akan dipindahkan menuju direktori yang ditentukan.
\item Pada pengiriman dan pemindahan \textit{file} dilakukan validasi sesuai dengan ketentuan yang ada.
\item Setelah \textit{file} diterima akan dikeluarkan pesan berhasil.
\end{enumerate}
Perangkat lunak akan menerima \textit{file} dari \textit{form} yang nantinya akan dilakukan validasi pada \textit{controller}. Kode \ref{kode:ci4fileuploadview} menunjukan contoh \textit{form} untuk melakukan pengunggahan \textit{file}.
\begin{lstlisting}[language=PHP, caption=Contoh \textit{form} untuk melakukan pengunggahan \textit{file}. ,label=kode:ci4fileuploadview]
<!DOCTYPE html>
<html lang="en">
<head>
    <title>Upload Form</title>
</head>
<body>

<?php foreach ($errors as $error): ?>
    <li><?= esc($error) ?></li>
<?php endforeach ?>

<?= form_open_multipart('upload/upload') ?>
    <input type="file" name="userfile" size="20">
    <br><br>
    <input type="submit" value="upload">
</form>

</body>
</html>
\end{lstlisting}
Kode \ref{kode:ci4fileuploadview} dibentuk menggunakan \textit{form helper} dan dapat mengembalikan pesan apabila terdapat \textit{error}. Setelah dilakukan penerimaan \textit{file}, perangkat lunak akan mengirimkan \textit{file} menuju \textit{controller} untuk dilakukan validasi dan penyimpanan. Kode \ref{kode:ci4fileuploadcontroller} menunjukan contoh \textit{controller} untuk melakukan validasi dan penyimpanan.
\begin{lstlisting}[language=PHP, caption=Contoh kode \textit{controller} untuk melakukan validasi dan penyimpanan. ,label=kode:ci4fileuploadcontroller]
<?php

namespace App\Controllers;

use CodeIgniter\Files\File;

class Upload extends BaseController
{
    protected $helpers = ['form'];

    public function index()
    {
        return view('upload_form', ['errors' => []]);
    }

    public function upload()
    {
        $validationRule = [
            'userfile' => [
                'label' => 'Image File',
                'rules' => [
                    'uploaded[userfile]',
                    'is_image[userfile]',
                    'mime_in[userfile,image/jpg,image/jpeg,image/gif,image/png,image/webp]',
                    'max_size[userfile,100]',
                    'max_dims[userfile,1024,768]',
                ],
            ],
        ];
        if (! $this->validate($validationRule)) {
            $data = ['errors' => $this->validator->getErrors()];

            return view('upload_form', $data);
        }

        $img = $this->request->getFile('userfile');

        if (! $img->hasMoved()) {
            $filepath = WRITEPATH . 'uploads/' . $img->store();

            $data = ['uploaded_fileinfo' => new File($filepath)];

            return view('upload_success', $data);
        }

        $data = ['errors' => 'The file has already been moved.'];

        return view('upload_form', $data);
    }
}
\end{lstlisting}
Kode \ref{kode:ci4fileuploadcontroller} terdapat dua buah fungsi sebagai berikut:
\begin{itemize}
\item \verb|index()| berfungsi untuk mengembalikan \textit{view} bernama \texttt{upload\_form}
\item \verb|upload()| berfungsi dalam memberikan aturan ukuran, dimensi, dan jenis \textit{file} untuk melakukan validasi. Setelah validasi berhasil fungsi ini melakukan penyimpanan pada direktori \texttt{uploads} dan mengembalikan halaman \textit{view}.
\end{itemize}

\subsubsection{\textit{Working with URIs}}
\textit{CodeIgniter 4} menyediakan solusi berbasis \textit{object oriented} untuk bekerja dengan URI. \textit{Library} ini simpel dan memastikan struktur selalu benar tanpa melihat URI yang kompleks. Berikut merupakan sintaks untuk membentuk sebuah \textit{URI}:
\begin{center}
	\verb|$uri = new \CodeIgniter\HTTP\URI('http://www.example.com/some/path');|
\end{center}
Sintaks diatas akan membentuk URI dengan nama \texttt{example.com} dengan beberapa \textit{segment} bernama \texttt{some} dan \texttt{path}. URI yang sudah dibentuk selanjutnya dapat diambil menggunakan sintaks berikut:
\begin{center}
	\verb|$uri = $this->request->getUri();|
\end{center}
Sintaks diatas akan mengambil URI yang terdapat pada halaman tersebut. Selain dapat mengambil URI, \textit{library} ini juga dapat mengambil \textit{segment} yang terdapat pada \textit{URI}. Berikut merupakan sintaks untuk mengambil \textit{segement} yang terdapat pada \textit{URI}.
\begin{center}
	\verb|$segments = $uri->getSegments();|
\end{center}
Sintaks diatas akan mengambil \textit{segment} yang tersedia pada \textit{URI} dan menyimpannya dalam sebuah \textit{array} berisikan \textit{value} dari setiap \textit{segment} yang tersedia.

\subsubsection{\textit{Validation}}
\textit{CodeIgniter 4} menyediakan \textit{library} untuk melakukan validasi terhadap data yang dikirimkan oleh pengguna. Data yang divalidasi dapat diberikan aturan-aturan sesuai dengan konfigurasi pengguna. Validasi akan melalui data yang dikirimkan oleh \textit{form}. Kode \ref{kode:ci4validationform} menunjukan contoh \textit{form} yang digunakan untuk mengirimkan data menuju \textit{controller}.

\begin{lstlisting}[language=PHP, caption=Contoh kode untuk melakukan pengumpulan data. ,label=kode:ci4validationform]
<html>
<head>
    <title>My Form</title>
</head>
<body>

    <?= validation_list_errors() ?>

    <?= form_open('form') ?>

        <h5>Username</h5>
        <input type="text" name="username" value="<?= set_value('username') ?>" size="50">

        <h5>Password</h5>
        <input type="text" name="password" value="<?= set_value('password') ?>" size="50">

        <h5>Password Confirm</h5>
        <input type="text" name="passconf" value="<?= set_value('passconf') ?>" size="50">

        <h5>Email Address</h5>
        <input type="text" name="email" value="<?= set_value('email') ?>" size="50">

        <div><input type="submit" value="Submit"></div>

    <?= form_close() ?>

</body>
</html>
\end{lstlisting}

\textit{Form} yang dibentuk dapat mengembalikan \textit{error} menggunakan sintaks \texttt{validation\_list\_errors()}. Selanjutnya akan digunakan fungsi \texttt{form\_open} untuk membuka \textit{tag form} sesuai dengan URL fungsi \textit{controller} yang sudah dibentuk. Setiap input akan diberikan \textit{name} untuk mengambil data pada \textit{controller}. Setelah \textit{tag form} selesai maka akan ditutup dengan sintaks \texttt{form\_close}. Setelah itu data-data yang sudah dimasukan akan dikirimkan menuju \textit{controller} seperti yang ditunjukan pada kode \ref{kode:ci4validationcontroller}.

\begin{lstlisting}[language=PHP, caption=Contoh kode untuk melakukan validasi data yang sudah dikumpulkan. ,label=kode:ci4validationcontroller]
<?php

namespace App\Controllers;

class Form extends BaseController
{
    protected $helpers = ['form'];

    public function index()
    {
        if (! $this->request->is('post')) {
            return view('signup');
        }

        $rules = [];

        if (! $this->validate($rules)) {
            return view('signup');
        }

        // If you want to get the validated data.
        $validData = $this->validator->getValidated();

        return view('success');
    }
}
\end{lstlisting}
Data-data yang sudah diberikan oleh pengguna akan diambil dan divalidasi menggunakan \textit{controller} pada Kode \ref{kode:ci4validationcontroller}. Sintaks \verb|if (! $this->request->is('post'))| akan melakukan pengecekan apakah \textit{request} yang diberikan berupa \textit{post} atau tidak. Selanjutnya dapat ditentukan aturan pada variabel \texttt{rules} yang nantinya dilakukan validasi menggunakan fungsi \texttt{validate}. Fungsi \texttt{validate} akan mengecek data yang diberikan dan menentukan apakah sudah sesuai dengan aturan yang ada atau belum. Kode \ref{kode:ci4validationrules} menunjukan contoh pembentukan aturan sesuai dengan \textit{name} setiap \textit{input form}.

\begin{lstlisting}[language=PHP, caption=Contoh kode untuk menetapkan aturan untuk validasi data yang sudah dikumpulkan. ,label=kode:ci4validationrules]
$rules = [
    'username' => 'required|max_length[30]',
    'password' => 'required|max_length[255]|min_length[10]',
    'passconf' => 'required|max_length[255]|matches[password]',
    'email'    => 'required|max_length[254]|valid_email',
];
\end{lstlisting}

Kode \ref{kode:ci4validationrules} melakukan pengecekan terhadap setiap \textit{input form} sesuai dengan aturan yang telah ditetapkan. Aturan-aturan tersebut dapat diganti sesuai dengan kebutuhan dari pengguna. Selain menggunakan aturan yang disediakan \textit{CodeIgniter 4}, pengguna dapat membentuk aturannya sendiri pada file \textit{Validation.php}. Kode \ref{kode:ci4validationrulesmanual} menunjukan contoh aturan yang dibentuk secara manual. Aturan yang dibentuk secara manual dapat digunakan sama seperti penggunaan aturan lainnya.

\begin{lstlisting}[language=PHP, caption=Contoh kode pembentukan aturan secara manual pada file \texttt{Validation.php}. ,label=kode:ci4validationrulesmanual]
<?php

class MyRules
{
    public function even($value): bool
    {
        return (int) $value % 2 === 0;
    }
}
\end{lstlisting}

\subsection{\textit{Helpers}}
\textit{Helpers} merupakan fitur pada \textit{CodeIgniter 4} yang menyediakan beberapa fungsi untuk pengguna dalam membangun aplikasi \textit{website}. \textit{Helpers} dapat dimuat oleh pengguna menggunakan sintaks sebagai berikut:

\begin{center}
\verb|<?php|

\verb|helper('helpers_name');|
\end{center}
Setelah dilakukan pemanggilan, pengguna dapat memakai fungsi-fungsi yang disediakan sesuai dengan \textit{helpers} yang digunakan. Fungsi-fungsi itu antara lain adalah \textit{form}, \textit{array}, dan \textit{text}.

\section{Konversi CodeIgniter 3 ke CodeIgniter 4\cite{codeigniter:23:ci4}}
\label{sec:konversici3c4}
 
Konversi CodeIgniter 3 ke CodeIgniter 4 diperlukan penulisan ulang karena terdapat banyak implementasi yang berbeda. Konversi ke CodeIgniter 4 diawali dengan melakukan instalasi projek baru CodeIgniter 4. Instalasi dapat dilakukan dengan mengunduh file melalui situs resmi ataupun dapat dilakukan melalui \textit{composer}.


\subsection{Struktur Aplikasi}

Struktur direktori pada CodeIgniter 4 memiliki perubahan yang terdiri \textit{app, public, dan writable}. Direktori app merupakan perubahan dari direktori application dengan isi yang hampir sama dengan beberapa perubahan nama dan perpindahan direktori. Pada CodeIgniter 4 terdapat direktori \textit{public} yang bertujuan sebagai direktori akar pada aplikasi \textit{website}. Selanjutnya terdapat direktori \textit{writable} yang berisikan \textit{cache data, logs,} dan \textit{session data}.

\subsection{\textit{Routing}}

\textit{CodeIgniter 4} meletakan \textit{route} pada \textit{file} \verb|app\Config\Routes.php|. \textit{CodeIgniter 4} memiliki fitur \textit{auto routing} seperti pada \textit{CodeIgniter 3} namun, secara \textit{default} dimatikan. Fitur \textit{auto routing} memungkinkan untuk dinyalakan serupa dengan pada CodeIgniter 3 namun tidak direkomendasikan karena alasan \textit{security}.
 
\subsection{\textit{Model, View,} dan \textit{Controller}}
 
Struktur MVC pada \textit{CodeIgniter 4} berbeda dengan \textit{CodeIgniter 3} dimana terdapat perbedaan penyimpanan direktori untuk ketiga \textit{file} tersebut. Berikut merupakan penjelasan mengenai struktur MVC:

\subsubsection{\textit{Model}}
\textit{Model} pada \textit{CodeIgniter 4} terletak pada direktori \verb|app\Models|. Pembaharuan \textit{Model} dapat dilakukan melalui cara sebagai berikut:
\begin{enumerate}
\item Pengguna harus memindahkan seluruh \textit{file model} menuju direktori \verb|app/Models|
\item Pengguna harus menambahkan \verb|namespace App\Models;| setelah pembukaan \textit{tag} PHP.
\item Pengguna juga harus menambahkan \verb|use CodeIgniter\Model;| setelah kode diatas \texttt{namespace}.
\item Pengguna harus mengganti \verb|extends CI_Model| menjadi \verb|extends Model|.
\item Terakhir pemanggilan \textit{model} berubah dari sintaks \verb|$this->load->model('x');| menjadi \verb|$this->x = new X();|.
\end{enumerate}
 
\subsubsection{\textit{View}}
\textit{View} pada CodeIgniter 4 terdapat di \verb|app\Views| dengan sintaks yang harus diubah. Pembaharuan \textit{view}dapat dilakukan dengan cara sebagai berikut:
\begin{enumerate}
\item Pengguna perlu memindahkan seluruh \textit{file views} menuju \verb|app/Views|
\item Pengguna perlu mengubah sintaks:
\begin{center}
	\verb|$this->load->view('directory_name/file_name')|
\end{center}
menjadi sintaks berikut:
\begin{center}
	\verb|return view('directory_name/file_name');|
\end{center}
\item Pengguna juga perlu mengubah sintaks:
\begin{center}
	\verb|$content = $this->load->view('file', $data, TRUE);|
\end{center}	
	 menjadi sintaks berikut:
\begin{center}
	\verb|$content = view('file', $data);|.
\end{center}
\item Pada \textit{file views} pengguna dapat mengubah sintaks \verb|<?php echo $title; ?>| menjadi sintaks \verb|<?= $title ?>|.

\item Pengguna juga perlu menghapus apabila terdapat sintaks \texttt{defined('BASEPATH') OR exit('No direct script access allowed');}.
\end{enumerate}
 
\subsubsection{\textit{Controller}}
\textit{Controller} pada CodeIgniter 4 terdapat di \verb|app\Controllers| dan diperlukan beberapa perubahan. Pengguna dapat melakukan pembaharuan \textit{controller} menggunakan cara sebagai berikut:
\begin{enumerate}
\item Pengguna harus memindahkan seluruh \textit{file controller} menuju \verb|app/Controllers|.
\item Pengguna juga harus menambahkan sintaks \verb|namespace App\Controllers;| setelah pembukaan \textit{tag PHP}.
\item Selanjutnya pengguna harus mengubah \verb|extends CI_Controller| menjadi \verb|extends BaseController|.
\item Pengguna juga harus menghapus apabila terdapat baris \texttt{defined('BASEPATH') OR exit('No direct script access allowed');}.
\end{enumerate}
 
\subsection{\textit{Class Loading}}
Pada \textit{CodeIgniter 4} sudah tidak terdapat \textit{superobject} dengan komponen \textit{framework} yang terinjeksi sebagai properti pada \textit{controller}. Kelas yang sudah dibentuk akan diinisiasikan di tempat yang membutuhkan dan komponen \textit{framework} akan diatur oleh \textit{service}. \textit{Autoloader} pada \textit{CodeIgniter 4} secara otomatis menangani lokasi kelas dengan standar \textit{PSR-4} di dalam direktori \textit{App}.
 
\subsection{\textit{Configuration}}

\textit{File configuration} CodeIgniter 4 terdapat pada \verb|app\Config| dengan penulisan sedikit berbeda dengan versi sebelumnya. Penulisan berubah dari yang sebelumnya menggunakan \textit{array} akan berubah menjadi menggunakan variabel. Pengguna hanya perlu melakukan pemindahan data menuju CodeIgniter 4 dan apabila menggunakan \textit{file config custom} maka, diperlukan penulisan ulang pada direktori Config dengan melakukan \textit{extend} pada \verb|CodeIgniter\Config\BaseConfig|. Beberapa konfigurasi juga akan dipindahkan menuju file \texttt{.env}.

\subsection{\textit{Database}}

Penggunaan \textit{database} pada CodeIgniter 4 hanya berubah sedikit dibandingkan dengan versi sebelumnya. Pengguna dapat melakukan pembaharuan menggunakan cara sebagai berikut:
\begin{enumerate}
	\item Memindahkan data-data kredensial menuju \textit{file} \verb|app\Config\Database.php|
	\item Mengubah sintaks untuk memuat \textit{database} diubah menjadi \verb|$db = db_connect();|
	\item Penggunaan beberapa \textit{databaes} diubah menggunakan sintaks \verb|$db = db_connect('group_name');|
	\item Mengubah seluruh \textit{query} dari \verb|$this->db| menjadi \verb|$db|
	\item Mengubah sintaks \textit{query} seperti \verb|$query->result();| menjadi \verb|$query->getResult();|
	\item Selain itu, terdapat pembaharuan \textit{Query Builder Class} yang harus di inisiasi menggunakan sintaks \texttt{\$builder = \$db->table('mytable');}. Kelas ini dapat dipakai untuk menjalankan \textit{query} menggunakan sintaks \verb|$builder->get();|
\end{enumerate}

\subsection{\textit{Migrations}}

Perubahan perlu dilakukan pada nama \textit{file} menjadi nama dengan cap waktu. Selanjutnya dilakukan penghapusan kode \verb|defined('BASEPATH') OR exit('No direct script access allowed');| dan menambahkan dua buah kode setelah membuka tag PHP seperti yang ditunjukan pada Kode \ref{kode:convmigration}.
\begin{lstlisting}[language=PHP, caption=Penambahan kode pada \textit{file migration}. ,label=kode:convmigration]
\item \verb|namespace App\Database\Migrations;|
	\item \texttt{use CodeIgniter/Database/Migration;}
\end{lstlisting}
Setelah itu, \verb|extends CI_Migration| diubah menjadi \verb|extends Migration|. Terakhir, terdapat perubahan pada nama metode \textit{Forge} dari yang sebelumnya bernama \verb|$this->dbforge->add_field| menjadi menggunakan \textit{camelCase} \verb|$this->forge->addField|.

\subsection{\textit{Routing}}

Pengguna dapat melakukan pembaharuan \textit{routing} dengan cara sebagai berikut:
\begin{enumerate}
\item Pengguna dapat memakai \textit{Auto Routing} seperti pada \textit{CodeIgniter 3} dengan menyalakan \textit{Auto Routing(Legacy)}.
\item Terdapat perubahan dari \verb|(:any)| menjadi \verb|(:segment)|.
\item Pengguna juga harus mengubah sintaks pada \verb|app/Config/Routes.php| sebagai berikut:
	\begin{itemize}
	\item \verb|$route['journals'] = 'blogs';| menjadi \texttt{\$routes->add('journals', 'Blogs::index');} 
	Sintaks diatas berguna untuk memanggil fungsi \texttt{index} pada \textit{controller} \texttt{Blogs}.
	\item \verb|$route['product/(:any)'] = 'catalog/product_lookup'| menjadi sintaks 		\texttt{\$routes->add('product/(:segment)', 'Catalog::productLookup');}.
	\end{itemize}
\end{enumerate}

\subsection{\textit{Libraries}}
 
CodeIgniter 4 menyediakan \textit{library} untuk digunakan dan dapat diinstall apabila diperlukan. Pemanggilan \textit{library} berubah dari \verb|$this->load->library('x');| menjadi \verb|$this->x = new X();|. Terdapat beberapa \textit{library} yang harus di perbaharui dengan sedikit perubahan. Berikut merupakan beberapa \textit{libraries} yang terdapat pembaharuan:

\subsubsection{\textit{Emails}}

Perubahan \textit{email} hanya terdapat pada nama dari \textit{method} dan pemanggilan \textit{library email}. Pemanggilan  \textit{library} berubah dari \verb|$this->load->library('email);| menjadi \verb|$email = service('email');| dan selanjutnya perlu dilakukan perubahan pada semua \verb|$this->email| menjadi \verb|$email|. Selanjutnya beberapa pemanggilan \textit{method} berubah dengan tambahan \textit{set} didepannya seperti \textit{from} menjadi \textit{setFrom}.

\subsubsection{\textit{Working with Uploaded Files}}

Terdapat banyak perubahan dimana pada \textit{CodeIgniter 4} pengguna dapat mengecek apakah \textit{file} telah terunggah tanpa \textit{error}. \textit{Library} ini juga mempermudah pengguna untuk melakukan penyimpanan \textit{file}. Pada CodeIgniter 4 melakukan akses pada \textit{uploaded file} dengan nama \texttt{userfile} dilakukan dengan sintaks sebagai berikut:
\begin{center}
\verb|$file = $this->request->getFile('userfile')|
\end{center} selanjutnya dapat dilakukan validasi dengan cara sebagai berikut:
\begin{center}
\verb|$file->isValid()|
\end{center} 
\textit{File} yang sudah berhasil diunggah dapat disimpan dengan sintaks berikut:
\begin{center}
\verb|$path = $this->request->getFile('userfile')->store('head_img/', 'user_name.jpg');|.
\end{center} 

Sintaks diatas akan mengambil file dengan atribut nama \texttt{userfile} dan menyimpannya pada direktori \texttt{head\_img} dengan nama file \texttt{user\_name.jpg}.

\subsubsection{\textit{HTML Tables}}

Tidak terdapat banyak perubahan pada \textit{framework} versi terbaru hanya perubahan pada nama \textit{method} dan pemanggilan \textit{library}. Perubahan pemanggilan \textit{library} dari \verb|$this->load->library('table');| menjadi \verb|$table = new \CodeIgniter\View\Table();| dan perlu dilakukan perubahan setiap \verb|$this->table| menjadi \verb|$table|. Selain itu, terdapat bebera perubahan pada penamaan \textit{method} dari \textit{underscored} menjadi \textit{camelCase} seperti \texttt{set\_heading} menjadi \texttt{setHeading}.


\subsubsection{Localization}

\textit{CodeIgniter 4} mengembalikan \textit{file} bahasa menjadi \textit{array} sehingga perlu dilakukan beberapa perubahan. Pertama, perlu dilakukan konfigurasi \textit{default language} 
pada perangkat lunak. Selanjutnya melakukan pemindahan \textit{file} bahasa pada \textit{CodeIgniter 3} menuju \verb|app\Language\<locale>|. \textit{File-file} bahasa \textit{CodeIgniter 3} perlu dilakukan penghapusan semua \textit{loader} \verb|$this->lang->load($file, $lang);|. Terakhir mengubah \textit{method} pemanggilan bahasa dari \verb|$this->lang->line('error_email_missing')| menjadi \verb|echo lang('Errors.errorEmailMissing');|
 

\subsubsection{\textit{Validations}}
Pengguna dapat melakukan pembaharuan pada \textit{validations} melalui cara sebagai berikut:
\begin{enumerate}
\item Pengguna harus mengubah kode pada \textit{view} dari \verb|<?php echo validation_errors(); ?>| menjadi \verb|<?= validation_list_errors() ?>|
\item Pengguna perlu mengubah beberapa kode pada \textit{controller} sebagai berikut:
	\begin{itemize}
		\item \verb|$this->load->helper(array('form', 'url'));| menjadi \verb|helper(['form', 'url']);|
		\item Pengguna perlu menghapus kode \verb|$this->load->library('form_validation');|
		\item \verb|if ($this->form_validation->run() == FALSE)| menjadi \verb|if (!$this->validate([]))|
		\item \verb|$this->load->view('myform');| \\ menjadi sintaks  \verb|return view('myform', ['validation' => $this->validator,]);|
	\end{itemize}
	\item Pengguna juga perlu mengubah kode seperti yang ditunjukan pada Kode \ref{kode:ci4validations} untuk melakukan validasi \textit{form}.
	\begin{lstlisting}[language=PHP, caption=Perubahan kode untuk melakukan validasi. ,label=kode:ci4validations]
<?php

$isValid = $this->validate([
    'name'  => 'required|min_length[3]',
    'email' => 'required|valid_email',
    'phone' => 'required|numeric|max_length[10]',
]);
\end{lstlisting}
\end{enumerate}

\subsection{\textit{Helpers}}
 
\textit{Helpers} tidak terdapat banyak perubahan namun, beberapa \textit{helpers} pada \textit{CodeIgniter 3} tidak terdapat pada \textit{CodeIgniter 4} sehingga perlu perubahan pada implementasi fungsinya. \textit{Helpers} dapat di dimuat secara otomatis menggunakan \verb|app\Config\Autoload.php|

\iffalse
\subsection{\textit{Events}}
\textit{Events} merupakan pembaharuan dari \textit{Hooks}. Pengguna harus mengubah
\begin{center}
	\verb|$hook['post_controller_constructor']|
\end{center} 
menjadi 
\begin{center} \verb|Events::on('post_controller_constructor', ['MyClass', 'MyFunction']);|
\end{center}
Dan menambahkan \textit{namespace} \verb|CodeIgniter\Events\Events;|. 
\fi

\subsection{\textit{Framework}}
Pengguna tidak membutuhkan direktori \textit{core} dan tidak membutuhkan kelas \verb|MY_X| pada direktori \textit{libraries} untuk memperpanjang atau mengganti potongan CI4. Pengguna dapat membangun kelas dimanapun dan menambahkan metode pada \verb|app/Config/Services.php|.