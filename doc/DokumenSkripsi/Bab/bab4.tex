\chapter{Perancangan}
\label{chap:perancangan}
Bab ini membahas perancangan untuk seluruh implementasi \textit{SharIF Judge} pada \textit{CodeIgniter 4}. Seluruh perancangan dilakukan dengan cara membandingkan kedua kode pada \textit{SharIF Judge} berbasis \textit{CodeIgniter 3} dan \textit{SharIF Judge} berbasis \textit{CodeIgniter 4}.

\section{Perancangan Perubahan Struktur Aplikasi}
Struktur aplikasi \textit{SharIF Judge} dipindahkan seperti pemetaan pada bab \ref{chap:analisis} gambar \ref{fig:dirMapping}. Pemindahan dilakukan dengan penyesuaian dan perubahan beberapa fungsi yang digunakan. Berikut merupakan perancangan struktur baru pada \textit{CodeIgniter 4}:
\subsection{app/Config}
\subsubsection{\texttt{app/Config/App.php}}
\textit{File} ini tidak digunakan sehingga dibiarkan berisi sintaks \textit{default}. Seluruh data akan dipindahkan menuju \textit{file} \texttt{.env}. Data-data yang dipindahkan antara lain adalah \texttt{baseURL}.
\subsubsection{\texttt{app/Config/Autoload.php}}
\textit{File} ini tidak terdapat perubahan karena terdapat perbedaan cara penggunaan antara pada \textit{CodeIgniter 3} dan \textit{CodeIgniter 4}.
\subsubsection{\texttt{app/Config/Cache.php}}
\textit{File} ini tidak terdapat perubahan karena pada \textit{CodeIgniter 3} menggunakan konfigurasi \textit{default}.
\subsubsection{\texttt{app/Config/Constants.php}}
\textit{File} ini tidak terdapat perubahan namun terdapat pemindahan beberapa sintaks dari \texttt{constants.php}.
\subsubsection{\texttt{app/Config/Cookie.php}}
\textit{File} ini tidak terdapat perubahan karena pada \textit{CodeIgniter 3} menggunakan konfigurasi \textit{default}.
\subsubsection{\texttt{app/Config/Database.php}}
\textit{File} ini tidak digunakan sehingga dibiarkan berisi sintaks \textit{default}. Seluruh data akan dipindahkan menuju \textit{file} \texttt{.env}. Data-data yang dipindahkan antara lain adalah konfigurasi \textit{database} aplikasi.
\subsubsection{\texttt{app/Config/Email.php}}
\textit{File} ini tidak terdapat perubahan karena pada \textit{CodeIgniter 3} menggunakan konfigurasi \textit{default}.
\subsubsection{\texttt{app/Config/Encryption.php}}
\textit{File} ini terdapat perubahan pada \textit{encryption key} yang dipindahkan dari \texttt{config/config.php}.
\subsubsection{\texttt{app/Config/Filters.php}}
\textit{File} ini terdapat penambahan penamaan seluruh \textit{filters} yang telah dibentuk.
\subsubsection{\texttt{app/Config/Routes.php}}
\textit{File} ini terdapat penambahan manual dari \textit{routes} pada \textit{SharIF Judge}.
\subsubsection{\texttt{app/Config/Secrets.example.php}}
\textit{File} ini dipindahkan menuju direktori \texttt{Config} dengan perubahan \texttt{extends} dan penghapusan sintaks \texttt{defined}.
\subsubsection{\texttt{app/Config/Security.php}}
\textit{File} ini terdapat perubahan pada nama \texttt{token} dan nama \texttt{cookie}.
\subsubsection{\texttt{app/Config/Session.php}}
\textit{File} ini terdapat perubahan tempat penyimpanan \textit{session} dan nama \texttt{cookie}.
\subsubsection{\texttt{app/Config/Validation.php}}
\textit{File} ini berisikan seluruh aturan yang dibentuk secara manual oleh pengguna. Terdapat pemindahan seluruh aturan untuk melakukan validasi dan terdapat penambahan sintaks untuk memproses data.

\subsection{Controllers}
\texttt{Controller} dipindahkan seluruhnya dari direktori \texttt{application/controllers} menuju direktori \texttt{app/Controllers}. Terdapat penghapusan sintaks \texttt{defined} dan perubahan \texttt{extends}. Berikut merupakan perubahan yang terdapat pada setiap \textit{controller}:
\subsubsection{\texttt{Assignments.php}}
Berikut merupakan perubahan yang terdapat pada \textit{controller} \texttt{Assignments.php}:
\begin{itemize}
	\item Perubahan cara pemanggilan \textit{model} dan \textit{library} yang digunakan.
	\item Perubahan cara mengembalikan halaman \textit{view}.
	\item Perubahan sintaks pengembalian \textit{error}.
	\item Perubahan \textit{library Form\_validation} menjadi \textit{Validation}.
	\item Perubahan \textit{library ZipEncoding} dan \textit{UnZip} menjadi \textit{Zip Archive}.
	\item Perubahan \textit{library Input} menjadi \textit{Request}.
	\item Perubahan \textit{Upload} menjadi \textit{Working with uploaded files}.
	\item Pemindahan pengecekan aplikasi menuju \textit{Filters}.
\end{itemize}
\subsubsection{\texttt{BaseController}}
\textit{Controller} \texttt{BaseController.php} merupakan penambahan \textit{controller} baru pada \textit{CodeIgniter 4}. Berikut merupakan penambahan sintaks pada \textit{controller} ini:
\begin{itemize}
	\item Pemanggilan \textit{helpers}
	\item Pemanggilan \textit{library third party} seperti \textit{Parsedown} dan \textit{Twig}.
\end{itemize}

\subsubsection{\texttt{Dashboard.php}}
Berikut merupakan perubahan yang terdapat pada \textit{controller} \texttt{Dashboard.php}:
\begin{itemize}
	\item Perubahan cara pemanggilan \textit{model} dan \textit{library} yang digunakan.
	\item Perubahan cara mengembalikan halaman \textit{view}.
	\item Perubahan sintaks pengembalian \textit{error}.
	\item Pemindahan pengecekan aplikasi menuju \textit{Filters}.
\end{itemize}
\subsubsection{\texttt{Halloffame.php}}
Berikut merupakan perubahan yang terdapat pada \textit{controller} \texttt{Halloffame.php}:
\begin{itemize}
	\item Perubahan cara pemanggilan \textit{model} dan \textit{library} yang digunakan.
	\item Perubahan cara mengembalikan halaman \textit{view}.
	\item Perubahan sintaks pengembalian \textit{error}.
	\item Perubahan \textit{library Input} menjadi \textit{Request}.
	\item Pemindahan pengecekan aplikasi menuju \textit{Filters}.
\end{itemize}
\subsubsection{\texttt{Install.php}}
Berikut merupakan perubahan yang terdapat pada \textit{controller} \texttt{Install.php}:
\begin{itemize}
	\item Perubahan cara pemanggilan \textit{model} dan \textit{library} yang digunakan.
	\item Perubahan cara mengembalikan halaman \textit{view}.
	\item Perubahan sintaks pengembalian \textit{error}.
	\item Perubahan cara penggunaan \textit{query}.
	\item Perubahan \textit{library Form\_validation} menjadi \textit{Validation}.
	\item Pemindahan pengecekan aplikasi menuju \textit{Filters}.
\end{itemize}
\subsubsection{\texttt{Login.php}}
Berikut merupakan perubahan yang terdapat pada \textit{controller} \texttt{Login.php}:
\begin{itemize}
	\item Perubahan cara pemanggilan \textit{model} dan \textit{library} yang digunakan.
	\item Perubahan cara mengembalikan halaman \textit{view}.
	\item Perubahan sintaks pengembalian \textit{error}.
	\item Perubahan cara penggunaan \textit{library} \textit{session}.
	\item Perubahan \textit{library Form\_validation} menjadi \textit{Validation}.
\end{itemize}
\subsubsection{\texttt{Logs.php}}
Berikut merupakan perubahan yang terdapat pada \textit{controller} \texttt{Logs\_model}:
\begin{itemize}
	\item Perubahan cara pemanggilan \textit{model} dan \textit{library} yang digunakan.
	\item Perubahan cara mengembalikan halaman \textit{view}.
	\item Perubahan sintaks pengembalian \textit{error}.
	\item Pemindahan pengecekan aplikasi menuju \textit{Filters}.
\end{itemize}
\subsubsection{\texttt{Moss.php}}
Berikut merupakan fungsi-fungsi pada \textit{controller} \texttt{Moss.php}:
\begin{itemize}
	\item Perubahan cara pemanggilan \textit{model} dan \textit{library} yang digunakan.
	\item Perubahan cara mengembalikan halaman \textit{view}.
	\item Perubahan sintaks pengembalian \textit{error}.
	\item Perubahan \textit{library Input} menjadi \textit{Request}.
	\item Perubahan \textit{library Form\_validation} menjadi \textit{Validation}.
	\item Pemindahan pengecekan aplikasi menuju \textit{Filters}.
\end{itemize}
\subsubsection{\texttt{Notification.php}}
Berikut merupakan perubahan yang terdapat pada  \textit{controller} \texttt{Notification.php}:
\begin{itemize}
	\item Perubahan cara pemanggilan \textit{model} dan \textit{library} yang digunakan.
	\item Perubahan cara mengembalikan halaman \textit{view}.
	\item Perubahan sintaks pengembalian \textit{error}.
	\item Perubahan \textit{library Input} menjadi \textit{Request}.
	\item Perubahan \textit{library Form\_validation} menjadi \textit{Validation}.
	\item Pemindahan pengecekan aplikasi menuju \textit{Filters}.
\end{itemize}
\subsubsection{\texttt{Problems.php}}
Berikut merupakan perubahan yang terdapat pada \textit{controller} \texttt{Problems.php}:
\begin{itemize}
	\item Perubahan cara pemanggilan \textit{model} dan \textit{library} yang digunakan.
	\item Perubahan cara mengembalikan halaman \textit{view}.
	\item Perubahan sintaks pengembalian \textit{error}.
	\item Perubahan \textit{library Input} menjadi \textit{Request}.
	\item Perubahan \textit{library Form\_validation} menjadi \textit{Validation}.
	\item Pemindahan pengecekan aplikasi menuju \textit{Filters}.
\end{itemize}
\subsubsection{\texttt{Profile.php}}
Berikut merupakan perubahan yang terdapat pada \textit{controller} \texttt{Profile.php}:
\begin{itemize}
	\item Perubahan cara pemanggilan \textit{model} dan \textit{library} yang digunakan.
	\item Perubahan cara mengembalikan halaman \textit{view}.
	\item Perubahan sintaks pengembalian \textit{error}.
	\item Perubahan \textit{library Input} menjadi \textit{Request}.
	\item Perubahan \textit{library Form\_validation} menjadi \textit{Validation}.
	\item Pemindahan pengecekan aplikasi menuju \textit{Filters}.
\end{itemize}
\subsubsection{\texttt{Queue.php}}
Berikut merupakan perubahan yang terdapat pada \textit{controller} \texttt{Queue.php}
\begin{itemize}
	\item Perubahan cara pemanggilan \textit{model} dan \textit{library} yang digunakan.
	\item Perubahan cara mengembalikan halaman \textit{view}.
	\item Perubahan sintaks pengembalian \textit{error}.
	\item Pemindahan pengecekan aplikasi menuju \textit{Filters}.
\end{itemize}
\subsubsection{\texttt{Queueprocess.php}}
Berikut merupakan perubahan yang terdapat pada \textit{controller} \texttt{Queueprocess.php}:
\begin{itemize}
	\item Perubahan cara pemanggilan \textit{model} yang digunakan.
	\item Perubahan cara pengambilan data pada file \textit{config}.
	\item Pemindahan pengecekan aplikasi menuju \textit{Filters}.
\end{itemize}
\subsubsection{\texttt{Rejudge.php}}
Berikut merupakan perubahan yang terdapat pada \textit{controller} \texttt{Rejudge.php}:
\begin{itemize}
	\item Perubahan cara pemanggilan \textit{model} dan \textit{library} yang digunakan.
	\item Perubahan cara mengembalikan halaman \textit{view}.
	\item Perubahan \textit{library Input} menjadi \textit{Request}.
	\item Perubahan \textit{library Form\_validation} menjadi \textit{Validation}.
	\item Perubahan sintaks pengembalian \textit{error}.
	\item Pemindahan pengecekan aplikasi menuju \textit{Filters}.
\end{itemize}
\subsubsection{\texttt{Scoreboard.php}}
Berikut merupakan perubahan yang terdapat pada \textit{controller} \texttt{Scoreboard.php}:
\begin{itemize}
	\item Perubahan cara pemanggilan \textit{model} yang digunakan.
	\item Perubahan cara mengembalikan halaman \textit{view}.
	\item Pemindahan pengecekan aplikasi menuju \textit{Filters}.
\end{itemize}
\subsubsection{\texttt{Server\_time.php}}
\textit{Controller} ini hanya terdapat pemindahan pengecekan aplikasi menuju \textit{Filters}.
\subsubsection{\texttt{Settings.php}}
Berikut merupakan perubahan yang terdapat pada \textit{controller} \texttt{Settings.php}:
\begin{itemize}
 	\item Perubahan cara pemanggilan \textit{model} dan \textit{library} yang digunakan.
	\item Perubahan cara mengembalikan halaman \textit{view}.
	\item Perubahan \textit{library Input} menjadi \textit{Request}.
	\item Perubahan \textit{library Form\_validation} menjadi \textit{Validation}.
	\item Pemindahan pengecekan aplikasi menuju \textit{Filters}.
\end{itemize}
\subsubsection{\texttt{Submission.php}}
Berikut merupakan perubahan yang terdapat pada \textit{controller} \texttt{Submission.php}:
\begin{itemize}
	\item Perubahan cara pemanggilan \textit{model} dan \textit{library} yang digunakan.
	\item Perubahan cara mengembalikan halaman \textit{view}.
	\item Perubahan \textit{library Input} menjadi \textit{Request}.
	\item Perubahan \textit{library Form\_validation} menjadi \textit{Validation}.
	\item Perubahan cara penggunaan \textit{library} \textit{session}.
	\item Perubahan cara penggunaan \textit{library URI}.
	\item Pemindahan pengecekan aplikasi menuju \textit{Filters}.
\end{itemize}
\subsubsection{\texttt{Submit.php}}
Berikut merupakan perubahan yang terdapat pada \textit{controller} \texttt{Submit.php}:
\begin{itemize}
	\item Perubahan cara pemanggilan \textit{model} dan \textit{library} yang digunakan.
	\item Perubahan cara mengembalikan halaman \textit{view}.
	\item Perubahan \textit{library Input} menjadi \textit{Request}.
	\item Perubahan \textit{library Form\_validation} menjadi \textit{Validation}.
	\item Pemindahan pengecekan aplikasi menuju \textit{Filters}.
\end{itemize}
\subsubsection{\texttt{Users.php}}
Berikut merupakan perubahan yang terdapat pada \textit{controller} \texttt{User.php}:
\begin{itemize}
	\item Perubahan cara pemanggilan \textit{model} dan \textit{library} yang digunakan.
	\item Perubahan cara mengembalikan halaman \textit{view}.
	\item Perubahan \textit{library Input} menjadi \textit{Request}.
	\item Perubahan \textit{library Form\_validation} menjadi \textit{Validation}.
	\item Perubahan cara penggunaan \textit{library} \textit{session}.
	\item Pemindahan pengecekan aplikasi menuju \textit{Filters}.
\end{itemize}

\subsection{Model}
\texttt{Model} dipindahkan seluruhnya dari direktori \texttt{application/models} menuju direktori \texttt{app/Models}. Berikut merupakan perubahan yang terdapat pada setiap \texttt{model}:
\subsubsection{\texttt{Assignment\_model.php}}
Berikut merupakan perubahan yang terdapat pada \texttt{Assignment\_model.php}:
\begin{itemize}
	\item Perubahan nama \textit{file} menjadi \texttt{AssignmentModel.php}.
	\item Perubahan cara pemanggilan \textit{model} dan \textit{library} lain yang digunakan.
	\item Perubahan sintaks dari \textit{snake case} menjadi \textit{camelCase}.
	\item Perubahan penggunaan \texttt{input} menjadi \texttt{request}.
	\item Penambahan sintaks \texttt{table} pada seluruh \textit{query}.
\end{itemize}

\subsubsection{\texttt{Hof\_model.php}}
Berikut merupakan perubahan yang terdapat pada \texttt{Hof\_model.php}:
\begin{itemize}
	\item Perubahan nama \textit{file} menjadi \texttt{HofModel.php}.
	\item Perubahan cara pemanggilan \textit{model} lain yang digunakan.
	\item Perubahan sintaks dari \textit{snake case} menjadi \textit{camelCase}.
\end{itemize}

\subsubsection{\texttt{Logs\_model.php}}
Berikut merupakan perubahan yang terdapat pada \texttt{Logs\_model.php}:
\begin{itemize}
	\item Perubahan nama \textit{file} menjadi \texttt{LogsModel.php}.
	\item Penambahan sintaks \texttt{table} pada seluruh \textit{query}.
	\item Perubahan sintaks dari \textit{snake case} menjadi \textit{camelCase}.
\end{itemize}
\subsubsection{\texttt{Notifications\_model.php}}
Berikut merupakan perubahan yang terdapat pada \texttt{Notifications\_model.php}:
\begin{itemize}
	\item Perubahan nama \textit{file} menjadi \texttt{NotificationsModel.php}.
	\item Penambahan sintaks \texttt{table} pada seluruh \textit{query}.
	\item Perubahan sintaks dari \textit{snake case} menjadi \textit{camelCase}.
\end{itemize}
\subsubsection{\texttt{Queue\_model.php}}
Berikut merupakan perubahan yang terdapat pada \texttt{Queue\_model.php}:
\begin{itemize}
	\item Perubahan nama \textit{file} menjadi \texttt{QueueModel.php}.
	\item Penambahan sintaks \texttt{table} pada seluruh \textit{query}.
	\item Perubahan cara pemanggilan \textit{model} lain yang digunakan.
	\item Perubahan sintaks dari \textit{snake case} menjadi \textit{camelCase}.
\end{itemize}
\subsubsection{\texttt{Scoreboard\_model.php}}
Berikut merupakan perubahan yang terdapat pada \texttt{Scoreboard\_model.php}:
\begin{itemize}
	\item Perubahan nama \textit{file} menjadi \texttt{ScoreboardModel.php}.
	\item Penambahan sintaks \texttt{table} pada seluruh \textit{query}.
	\item Perubahan cara pemanggilan \textit{model} lain yang digunakan.
	\item Perubahan sintaks dari \textit{snake case} menjadi \textit{camelCase}.
	\item Perubahan sintaks pengembalian halaman.
\end{itemize}
\subsubsection{\texttt{Settings\_model.php}}
Berikut merupakan perubahan yang terdapat pada \texttt{Settings\_model.php}:
\begin{itemize}
	\item Perubahan nama \textit{file} menjadi \texttt{SettingsModel.php}.
	\item Penambahan sintaks \texttt{table} pada seluruh \textit{query}.
	\item Perubahan sintaks dari \textit{snake case} menjadi \textit{camelCase}.
\end{itemize}
\subsubsection{\texttt{Submit\_model.php}}
Berikut merupakan perubahan yang terdapat pada \texttt{Submit\_model.php}:
\begin{itemize}
	\item Perubahan nama \textit{file} menjadi \texttt{SubmitModel.php}.
	\item Penambahan sintaks \texttt{table} pada seluruh \textit{query}.
	\item Perubahan cara pemanggilan \textit{model} lain yang digunakan.
	\item Perubahan sintaks dari \textit{snake case} menjadi \textit{camelCase}.
\end{itemize}
\subsubsection{\texttt{User\_model.php}}
Berikut merupakan perubahan yang terdapat pada \texttt{User\_model.php}:
\begin{itemize}
	\item Perubahan nama \textit{file} menjadi \texttt{UserModel.php}.
	\item Penambahan sintaks \texttt{table} pada seluruh \textit{query}.
	\item Perubahan cara pemanggilan \textit{model} dan \textit{library} lain yang digunakan.
	\item Perubahan sintaks dari \textit{snake case} menjadi \textit{camelCase}.
	\item Perubahan \textit{library} \texttt{HashPassword} menjadi \texttt{password\_hash}.
\end{itemize}
\subsubsection{\texttt{User.php}}
Berikut merupakan perubahan yang terdapat pada \texttt{User.php}:
\begin{itemize}
	\item Perubahan cara pemanggilan \textit{library} yang digunakan.
	\item Penambahan sintaks \texttt{table} pada seluruh \textit{query}.
	\item Perubahan sintaks dari \textit{snake case} menjadi \textit{camelCase}.
\end{itemize}

\subsection{View}
\texttt{View} dipindahkan seluruhnya dari direktori \texttt{application/views} menuju direktori \texttt{app/Views}. \texttt{View} berubah dari yang sebelumnya menggunakan \textit{template engine} bernama Twig menjadi menggunakan PHP. Seluruh \textit{extension} dari \textit{file View} berubah dari \texttt{.twig} menjadi \texttt{.php}. Selain itu terdapat perubahan beberapa \textit{delimiters}. \textit{Delimiters} yang menggunakan sintaks PHP berubah menjadi \texttt{<?php ?>}. \textit{Delimiters} yang menggunakan fitur pada Twig berubah menjadi \texttt{<?= ?>}. Terdapat juga perubahan fungsi pada \textit{template engine} Twig menjadi PHP dan perubahan cara pengambilan data dari \textit{controller}.

\subsection{\textit{Filters}}
\textit{Filters} merupakan fitur baru pada \textit{CodeIgniter 4}. Fitur ini memungkinan untuk melakukan pengecekan sebelum dan setelah \textit{controller} dijalankan. Fitur ini digunakan karena \textit{controller} pada \textit{CodeIgniter 4} tidak dapat mengembalikan apapun sehingga seluruh pengecekan pada \textit{controller}  dipindahkan menuju fitur ini. Berikut merupakan perancangan \textit{filter} yang dibentuk:
\begin{itemize}
	\item \texttt{CheckCLI.php} \\
	\textit{Filters} ini merupakan pemindahan dari \textit{controller} \texttt{Queueprocess.php}.
	\item \texttt{CheckInstallAndLogin.php}\\
	\textit{Filters} ini merupakan pemindahan dari \textit{controller} \texttt{Dashboard.php}.
	\item \texttt{CheckLogin.php}\\
	\textit{Filters} ini merupakan pemindahan dari \textit{controller} \texttt{Profile.php}, \texttt{Logs.php}, \texttt{Notification.php}, \texttt{Assignments}, \texttt{Submit.php}, \texttt{Submissions.php}, dan \texttt{Problems.php}.
	\item \texttt{CheckLoginandCLI.php}\\
	\textit{Filters} ini merupakan pemindahan dari \textit{controller} \texttt{Dashboard.php} dan \texttt{Scoreboard.php}.
	\item \texttt{CheckLoginandisAjax.php}\\
	\textit{Filters} ini merupakan pemindahan dari \textit{controller} \texttt{Server\_time.php}.
	\item \texttt{CheckLoginandLevelAdmin.php}\\
	\textit{Filters} ini merupakan pemindahan dari \textit{controller} \texttt{Settings.php} dan \texttt{Users.php}.
	\item \texttt{CheckLoginandLevelHead.php}\\
	\textit{Filters} ini merupakan pemindahan dari \textit{controller} \texttt{Moss.php}, \texttt{Queue.php} dan \texttt{Rejudge.php}.
\end{itemize}

\subsection{\textit{Routing}}
\textit{Routing} pada \textit{SharIF Judge} berbasis \textit{CodeIgniter 4} didefinisikan secara manual pada seluruh \textit{controller} dan fungsinya.

\subsection{\textit{Libraries}}
Terdapat penghapusan \textit{library} \textit{Zip Encoding, UnZip, Twig,} dan \textit{Password\_hash}. \textit{Library} lainnya dipindahkan menuju direktori \texttt{Libraries} dengan pengapusan sintaks \texttt{defined} dan penambahan sintaks \texttt{namespace}. \textit{Library Twig} digunakan kembali untuk melakukan pengubahan format waktu dan tidak digunakan untuk menampilkan halaman.